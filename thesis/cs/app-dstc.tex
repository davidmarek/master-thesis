% !TEX root = thesis.tex
\chapter{Zpracování dialogů z DSTC}
\label{ap:dstc}

\section{Soubory}
\label{sec:dstcfiles}
Předvedeme použití vytvořené knihovny na datech z Dialog State Tracking Challenge.
Všechna potřebné data a zdrojové kódy jsou na přiloženém CD.
V adresáři \texttt{dstc/} je velké množství modulů a dalších adresářů:
\begin{itemize}
\item \texttt{dstcutils.py}, \texttt{jsonpath.py}, \texttt{ontology.py}, \texttt{pddiscrete.py}, \\\texttt{common.py}, \texttt{preprocessing.py} -- pomocné moduly pro práci s daty, vytvořené členy výzkumné skupiny pracující na projektu Vystadial,

\item \texttt{loader.py} -- modul pro načítání dat z formátu JSON~\cite{crockford2006application} do interní reprezentace, 

\item \texttt{system\_dm.py} -- skript pro zpracování dat používající vytvořenou knihovnu pro odhad stavu v jednotlivých dialozích.
Použitá datová sada, vstupní a~výstupní soubory, parametry grafického modelu a další parametry nutné pro výpočet je možné skriptu zadat buď pomocí přepínačů na příkazové řádce, anebo v konfiguračním souboru.

\item \texttt{test\_run.sh} -- bashový skript pro spuštění celé pipeline od zpracování dat až po zobrazení výsledků,

\item \texttt{test1.conf}, \texttt{test2.conf}, \texttt{test3.conf}, \texttt{test4.conf} -- konfigurační soubory pro spuštění skriptu \texttt{system\_dm.py} na testovací datové sady ze všech čtyř dialogových systémů.

\item V adresáři \texttt{dstc/bin/} je baseline dialogový systém a skripty pro zpracování výsledků od organizátorů soutěže:
	\begin{itemize}
	\item \texttt{baseline} -- baseline dialogový systém,
	\item \texttt{score} -- skript pro vyhodnocení výstupu soutěžního systému,
	\item \texttt{report} -- skript pro zobrazení výsledků soutěžního systému v čitelné podobě,
	\item \texttt{validate} -- skript pro validaci výstupu soutěžního systému,
	\item Pomocné moduly pro tyto skripty jsou v adresáři \texttt{dstc/lib/}.
	\end{itemize}

\item V adresáři \texttt{dstc/data/} jsou všechny 4 testovací datové sady, každá ve vlastním adresáři (\texttt{test1/}, \texttt{test2/}, \texttt{test3/}, \texttt{test4/}).
Datové sady jsou komprimované a je třeba je nejprve rozbalit.
Každá datová sada se pak skládá z podadresářů pro jednotlivé hovory.
Každý hovor se skládá ze dvou souborů:
	\begin{itemize}
	\item \texttt{dstc.log.json} -- záznam z dialogového systému obsahující všechny informace o dialogu ve formátu JSON,
	\item \texttt{dstc.labels.json} -- soubor obsahující správné hypotézy, tedy řešení úlohy. Tento soubor samozřejmě před ukončením soutěže v datech  chyběl.
	\end{itemize}

\item V adresáři \texttt{dstc/config/} jsou soubory obsahující cesty k jednotlivým dialogům pro každou testovací sadu.
\end{itemize}

\section{Konfigurace}

Skript \texttt{system\_dm.py} očekává řadu parametrů pro správné fungování.
Parametry je možné zadat na příkazové řádce, například:
\begin{verbatim}
$ python system_dm.py \
    --dataroot=data/ \
    --dataset=config/test1.sessions \
    --scorefile=track_test1.json
\end{verbatim}

Popřípadě je možné vytvořit konfigurační soubor obsahující pouze parametry:
\begin{verbatim}
--dataroot=data/
--dataset=config/test1.sessions
--scorefile=track_test1.json
\end{verbatim}
a tento soubor pak předat jako jediný parametr:
\begin{verbatim}
$ python system_dm.py --flagfile test1.conf
\end{verbatim}

Většina vlastností systému pro odhad stavu je konfigurovatelná, ale jen malý počet parametrů je povinný.
Představíme si ty nejdůležitější parametry:
\begin{itemize}
\item \emph{dataroot} -- povinný, adresář obsahující datové sady,
\item \emph{dataset} -- povinný, konfigurační soubor obsahující seznam cest k jednotlivým dialogům,
\item \emph{scorefile} -- povinný, název souboru, do kterého má být zapsán výsledek zpracování dialogů,
\item \emph{slu} (live / batch) -- přepínač, zda-li pracovat s hypotézami z live nebo batch dat,
\item \emph{remove\_inconsistent} -- mazat nekonzistentní hypotézy (např. v jedné hypotéze více hodnot pro jeden slot),
\item \emph{processes} -- systém podporuje zpracování na více jádrech. Pomocí tohoto přepínače je možné nastavit počet použitých jader. Pro jedno jádro systém přepne do ladícího módu, kdy si udržuje v paměti všechny dialogy. Nehodí se pro velké datové sady.
\item \emph{prob\_same} -- hodnota parametru $\theta_o$, tedy pravděpodobnost, že pozorování odpovídá skryté proměnné,
\item \emph{prob\_trans} -- hodnota parametru $\theta_t$, tedy pravděpodobnost, že se hodnota skryté proměnné nezmění v další obrátce.
\end{itemize}

Přepínače jsou implementovány pomocí knihovny \texttt{python-gflags}, je třeba ji mít nainstalovanou.

\section{Výpočet}

V předchozí sekci bylo ukázáno, jakým způsobem lze nastavit a spustit skript \texttt{system\_dm.py}, který provede odhad dialogového stavu.
Výsledkem je soubor ve formátu JSON, kde pro každý dialog a každou jeho obrátku jsou uloženy hypotézy pro každý slot.
V příkladu~\ref{ex:res} je znázorněn výstup pro jeden slot, v tomto případě se jedná o počáteční zastávku, kde hodnota \uv{shadyside} má pravděpodobnost přibližně $0.56$.

\begin{example}
\begin{Verbatim}[commandchars=\\\{\}]

\PY{n+nt}{\PYZdq{}from.neighborhood\PYZdq{}}\PY{p}{:} \PY{p}{\PYZob{}}
    \PY{n+nt}{\PYZdq{}hyps\PYZdq{}}\PY{p}{:} \PY{p}{[}
        \PY{p}{\PYZob{}}
	    \PY{n+nt}{\PYZdq{}score\PYZdq{}}\PY{p}{:} \PY{l+m+mf}{0.5637651968841264}\PY{p}{,}
            \PY{n+nt}{\PYZdq{}slots\PYZdq{}}\PY{p}{:} \PY{p}{\PYZob{}}
	        \PY{n+nt}{\PYZdq{}from.neighborhood\PYZdq{}}\PY{p}{:} \PY{l+s+s2}{\PYZdq{}shadyside\PYZdq{}}
            \PY{p}{\PYZcb{}}
        \PY{p}{\PYZcb{}}
    \PY{p}{]}
\PY{p}{\PYZcb{}}\PY{p}{,}
\end{Verbatim}
\caption{Část výsledku systému pro odhad stavu.}
\label{ex:res}
\end{example}

Po zpracování dat je třeba výstup vyhodnotit a následně výsledky zobrazit.
Všechny tři kroky je možné provést naráz použitím skriptu \texttt{test\_run.sh}:

\begin{verbatim}
$ ./test_run.sh test1.conf
\end{verbatim}

Ukázka výsledků je v příkladech~\ref{ex:res1}~a~\ref{ex:res2}. 
Ve výsledku jsou zobrazeny tři tabulky s výsledky. 
V první tabulce (\uv{schedule1}) jsou statistiky počítány po každé obrátce.
V druhé tabulce (\uv{schedule2}) jsou statistiky počítány pro slot pouze pokud se o něm v dané obrátce mluvilo.
Nakonec ve třetí tabulce (\uv{schedule3}) jsou statistiky počítány pouze na konci dialogu.

Jednotlivé sloupce reprezentují různé sloty, navíc je zde sloupec \emph{joint}, který hodnotí sdruženou informaci o všech slotech v dané obrátce, a sloupec \emph{all} obsahující průměrné skóre.

Řádky obsahují statistiky o slotech, zvláště zajímavé jsou:
\begin{itemize}
\item accuracy -- přesnost,
\item avgp -- průměrné skóre přiřazené správné hypotéze,
\item l2 -- Brier skóre.
\end{itemize}

Nakonec jsou zobrazeny i základní informace o celém výpočtu, například jaká datová sada byla zpracována, počet hovorů, počet obrátek, celkový čas a průměrný čas na jednu obrátku.

{
\footnotesize
\begin{example}
\begin{lstlisting}[language=c,frame=none]
----------------------------------------------------------------------------
                                    schedule1
----------------------------------------------------------------------------
               route from.d from.m from.n to.des to.mon to.nei  joint    all
            N  10085  10085  10085  10085  10085  10085  10085  10085  90765
     accuracy 0.8817 0.6969 0.9841 0.7771 0.8256 0.9845 0.7327 0.2642 0.8534
         avgp 0.8794 0.6779 0.9811 0.6765 0.8235 0.9828 0.7185 0.2465 0.8370
     hypcount 0.0773 0.2038 0.0103 0.2442 0.0435 0.0045 0.0353 0.4943 0.0729
           l2 0.1703 0.4542 0.0268 0.4575 0.2495 0.0243 0.3981 1.0655 0.2304
          mrr 0.8977 0.7239 0.9850 0.7838 0.8329 0.9846 0.7333 0.2888 0.8606
     nonempty 0.0682 0.1660 0.0103 0.2442 0.0367 0.0045 0.0348 0.4943 0.0669
  roc.v1_ca05 0.0000 0.0000 0.9841 0.0000 0.0000 0.9845 0.0000 0.0000 0.0000
  roc.v1_ca10 0.8617 0.0000 0.9841 0.0000 0.0000 0.9845 0.0000 0.0000 0.0000
  roc.v1_ca20 0.8817 0.0000 0.9841 0.0000 0.8256 0.9845 0.0000 0.0000 0.8534
   roc.v1_eer 0.8817 0.6969 0.9841 0.4288 0.8256 0.9845 0.7327 0.2642 0.8534
  roc.v2_ca05 0.0000 0.0000 0.0000 0.0000 0.0000 0.0000 0.0000 0.0000 0.0000
  roc.v2_ca10 0.0000 0.0000 0.0000 0.0000 0.0000 0.0000 0.0000 0.0000 0.0000
  roc.v2_ca20 0.0000 0.0000 0.0000 0.0000 0.0000 0.0000 0.0000 0.0000 0.0000


----------------------------------------------------------------------------
                                    schedule2
----------------------------------------------------------------------------
               route from.d from.m from.n to.des to.mon to.nei  joint    all
            N   6100   4837    926   2706   4070    904   2283   9974  27536
     accuracy 0.8590 0.6729 0.9471 0.6936 0.6784 0.9436 0.4717 0.2669 0.7309
         avgp 0.8589 0.6685 0.9449 0.6414 0.6773 0.9351 0.4496 0.2490 0.7212
     hypcount 0.0807 0.1784 0.0248 0.1778 0.0744 0.0243 0.0604 0.4927 0.0934
           l2 0.1994 0.4670 0.0780 0.5071 0.4562 0.0918 0.7784 1.0621 0.3939
          mrr 0.8769 0.6997 0.9514 0.7051 0.6907 0.9441 0.4733 0.2916 0.7443
     nonempty 0.0723 0.1333 0.0248 0.1778 0.0585 0.0243 0.0600 0.4927 0.0812
  roc.v1_ca05 0.0000 0.0000 0.9471 0.0000 0.0000 0.0000 0.0000 0.0000 0.0000
  roc.v1_ca10 0.0000 0.0000 0.9471 0.0000 0.0000 0.9436 0.0000 0.0000 0.0000
  roc.v1_ca20 0.8590 0.0000 0.9471 0.0000 0.0000 0.9436 0.0000 0.0000 0.0000
   roc.v1_eer 0.8590 0.6729 0.9471 0.6936 0.6784 0.9436 0.4717 0.2669 0.7309
  roc.v2_ca05 0.0000 0.0000 0.0000 0.0000 0.0000 0.0000 0.0000 0.0000 0.0000
  roc.v2_ca10 0.0000 0.0000 0.0000 0.0000 0.0000 0.0000 0.0000 0.0000 0.0000
  roc.v2_ca20 0.0000 0.0000 0.0000 0.0000 0.0000 0.0000 0.0000 0.0000 0.0000
\end{lstlisting}
\caption{Výsledky systému pro odhad stavu podle prvních dvou typů hodnocení.}
\label{ex:res1}
\end{example}

\begin{example}
\begin{lstlisting}[language=c,frame=none]
----------------------------------------------------------------------------
                                    schedule3
----------------------------------------------------------------------------
               route from.d from.m from.n to.des to.mon to.nei  joint    all
            N    897    874    422    767    767    354    620    901   5784
     accuracy 0.8863 0.6739 0.9645 0.7471 0.7249 0.9520 0.4048 0.1576 0.7076
         avgp 0.8827 0.6368 0.9496 0.5949 0.7159 0.9426 0.3581 0.1526 0.6692
     hypcount 0.0836 0.2574 0.0355 0.3494 0.0704 0.0198 0.1145 0.6071 0.1376
           l2 0.1656 0.5109 0.0713 0.5730 0.4012 0.0811 0.9077 1.1984 0.4673
          mrr 0.9036 0.7008 0.9656 0.7562 0.7347 0.9520 0.4073 0.1770 0.7184
     nonempty 0.0758 0.2082 0.0355 0.3494 0.0600 0.0198 0.1129 0.6071 0.1272
  roc.v1_ca05 0.0000 0.0000 0.9645 0.0000 0.0000 0.9520 0.0000 0.0000 0.0000
  roc.v1_ca10 0.8740 0.0000 0.9645 0.0000 0.0000 0.9520 0.0000 0.0000 0.0000
  roc.v1_ca20 0.8863 0.0000 0.9645 0.0000 0.0000 0.9520 0.0000 0.0000 0.0000
   roc.v1_eer 0.8863 0.6739 0.0664 0.5176 0.7249 0.9520 0.4048 0.1576 0.7076
  roc.v2_ca05 0.0000 0.0000 0.0000 0.0000 0.0000 0.0000 0.0000 0.0000 0.0000
  roc.v2_ca10 0.0000 0.0000 0.0000 0.0000 0.0000 0.0000 0.0000 0.0000 0.0000
  roc.v2_ca20 0.0000 0.0000 0.0000 0.0000 0.0000 0.0000 0.0000 0.0000 0.0000


----------------------------------------------------------------------------
                                    basic stats
----------------------------------------------------------------------------
             dataset : config/test1.sessions
      scorer_version : 1.7
            sessions : 715
     total_wall_time : 114.580698967
               turns : 10085
  wall_time_per_turn : 0.0113614971707
\end{lstlisting}
\caption{Výsledky systému pro odhad stavu podle třetího typu hodnocení a základní statistiky.}
\label{ex:res2}
\end{example}
}
