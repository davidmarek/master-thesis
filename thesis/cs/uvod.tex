% !TEX root = prace.tex
\chapter*{Úvod}
\addcontentsline{toc}{chapter}{Úvod}

Dialog je přirozený způsob dorozumívání a sdělování informací mezi lidmi.
Počítač, který by dokázal vést dialog s~uživatelem, byl vždy snem nejen příznivců vědecko-fantastické literatury.
Už pro první počítače vznikaly programy snažící se využít přirozenou řeč pro interakci s~uživatelem.
Jedním z~takových programů byla například Eliza \cite{weizenbaum1966eliza}, program jednající jako psychiatr.
Fungoval na principu rozpoznání textu pomocí gramatiky a následné transformace textu do promluv dle pravidel.
Avšak gramatiky a pravidlové systémy se ukázaly nedostačné pro praktické aplikace a tak se vývoj přesunul do statistických metod.

Z~počátku byla výkonnost a úspěšnost systémů pracujících s~přirozeným jazykem nízká, což vedlo ke skeptickým názorům na jejich budoucnost.
Asi téměř všichni si pamatují první pokusy o~diktování textu v~kancelářských aplikacích.
Systémy pro rozpoznání přirozeného jazyka se však stále zlepšovaly a dostaly se do fáze, kdy už je možné využít hlas jako vstup pro komplexní systémy.

Tato práce se zabývá doménově omezenými statistickými dialogovými systémy.
Dialogové systémy mohou vyhledávat spojení v~městské dopravě, hledat turistické informace, ovládat navigaci v~automobilu atd.
Nabízejí spoustu výhod, jsou dostupné pro nevidomé, mohou pracovat 24 hodin denně a mohou nahradit lidské operátory, což může potenciálně vést k~úsporám.

Dialogové systémy mají stále daleko k~dokonalosti.
Neumí se dobře vyrovnat se špatně rozpoznanými vstupy, šumem a nejasnostmi.
Na rozdíl od živých operátorů nedokáží improvizovat.
V~některých případech není uživatel spokojený s~dialogovým systémem jen proto, že nepochopil otázku a dialogový systém není schopen se zeptat jinak.

První problém se snaží řešit statistické dialogové systémy.
Pokud si nejsme jisti cílem uživatele, nabízí se řešení v~modelování uživatele pomocí pravděpodobnostního modelu.
Díky němu se pak můžeme vyrovnat s~chybami zapříčiněnými špatně rozpoznanými vstupy.
Tato práce se bude zabývat metodami pro efektivní výpočty ve statistických modelech aplikovatelných v~dialogových systémech.

\section*{Rozdělení práce}

Nejprve v~kapitole \ref{ch:kap1} popíšeme dialogové systémy a jejich jednotlivé součásti.
Blíže se budeme zabývat dialogovým stavem a jeho reprezentací pomocí Bayesovského přístupu.

Jako vhodnou strukturu pro reprezentaci dialogového stavu si představíme Bayesovské sítě v~kapitole \ref{ch:kap2}.
Ukážeme několik metod pro inferenci v~Bayesovských sítích.
Začneme jednoduchou, ale pomalou exaktní inferencí a přesuneme se až k~efektivní aproximativní inferenci pomocí algoritmu Loopy Belief Propagation.
Nakonec ukážeme metodu Expectation Propagation pro inferenci v~grafickém modelu s~aproximovanými faktory, která je zobecněním předchozích metod.

Pro využití v~dialogových systémech je možné buď nastavit pevné parametry (ať už ručně nastavené nebo naučené z~dat), anebo zvolit Bayesovský přístup i pro parametry.
Parametry pocházející z~Dirichletovského rozdělení jsou prezentovány v~kapitole \ref{ch:ep}.

V~kapitole \ref{ch:kap4} jsou popsány všechny implementované části a také příklady jejich použití.
V~této kapitole je také popsána soutěž Dialog State Tracking Challenge, do které byl přihlášen generativní systém pro odhad dialogového stavu používající popsanou knihovnu.
Tato soutěž testovala systémy pro odhad dialogového stavu na datech z~reálného dialogového systému.

\section*{Cíle}

\begin{enumerate}
\item Dialogové systémy často využívají pouze nejlepší hypotézu ze systému porozumění přirozené řeči.
    Většina ovšem umí vytvořit seznam $n$ nejlepších hypotéz.
    Tato práce si klade za cíl představit metody pro inferenci dialogového stavu v~dialogovém systému s~využitím více hypotéz.
    Představené metody budou založeny na reprezentaci dialogového stavu pomocí dynamických bayesovských sítí.
\item Bude představen algoritmus Loopy Belief Propagation a implementován pro využití v~reálných systémech pro odhad dialogového stavu.
\item Algoritmus bude otestován na datech z~reálného dialogového systému Let's Go! a porovnán s~dalšími účastníky soutěže Dialog State Tracking Challenge.
\item Důležitou částí algoritmů pro inferenci je určení pořadí, v~jakém má inference probíhat.
    Práce bude obsahovat implementaci několika strategií pro inferenci. Díky nim bude možné zvolit strategii pro efektivní inferenci v~různých druzích Bayesovských sítí (stromy, dynamické sítě, obecné grafy).
\item Nakonec se práce bude zabývat učením parametrů sítě a představí algoritmus Expectation Propagation, algoritmus pro nalezení aproximace pravděpodobnostního rozdělení.
\item Při většině reálných použití dochází k~aproximacím a úpravám modelu v~závislosti na problému tak, aby bylo možné inferenci provádět v~reálném čase.
Práce bude obsahovat implementaci frameworku, do kterého je možné jednoduše zasadit vlastní moduly pro aproximaci pravděpodobnostních rozdělení.
Ty pak budou používány algoritmem Expectation Propagation.
Jako příklad bude ukázán systém pro učení parametrů pravděpodobnostní distribuce v~generickém modelu pro reprezentaci dialogového stavu.
\end{enumerate}
