% !TEX root = prace.tex
\chapter{Implementace}
V předchozích kapitolách byly popsány teoretické základy nutné pro implementaci inference v bayesovských sítích.
V této kapitole bude popsána vytvořená knihovna pro dialogové systémy.
Celá knihovna se skládá z několika vrstev, každá z nich stojí na předchozí.
Nejnižší vrstva implementuje efektivní počítání s faktory.
Nad ní stojí vrstva reprezentující jednotlivé vrcholy ve faktor grafu.
Tato vrstva také obsahuje funkcionalitu pro počítání zpráv.
Nejvyšší vrstva se zaměřuje na samotnou propagaci zpráv.

\section{Diskrétní faktor}

Faktor je základním stavebním kamenem.
Operace s faktory musí být efektivní, pro výpočet jedné zprávy je potřeba několik násobení faktorů, následně je třeba je marginalizovat atd.
Faktory je třeba úsporně reprezentovat, každá zpráva, každé pravděpodobnostní rozhraní je samo o sobě faktor.

\subsection{Reprezentace faktorů}

Každý diskrétní faktor má seznam diskrétních proměnných, které tvoří jeho doménu.
Každá z těchto proměnných může mít jinou kardinalitu, některé proměnné jsou binární, jiné mají mnohem více hodnot.
Faktor je tedy ve své podstatě multidimenzionální tabulka.
Pro implementaci je tato tabulka zploštěná do jednoduchého pole.

Knihovna je napsaná v Pythonu a pro matematické operace používá knihovnu Numpy~\cite{oliphant-2006-guide}.
Pole pak využíváji implementaci z knihovny Numpy, díky které jsou matematické operace napsané v rychlejším jazyku (C, Fortran) než je Python a jsou navíc vektorizované.

Jak je tedy možné reprezentovat multidimenzionální tabulku jednodimenzionálním polem?
Proměnné jsou seřazené a pro zjednodušení můžeme předpokládat, že hodnoty proměnných jsou čísla z $\{0, \dots, n-1\}$, kde $n$ je kardinalita proměnné.
Každá hodnota v poli je pak hodnotou faktoru pro nějaké přiřazení hodnot proměnným, např. $(0, 1, 0)$.
Jednotlivé hodnoty jsou v poli seřazeny lexikograficky.
Pro každou proměnnou si pamatujeme její kardinalitu a také tzv. krok.
Krok určuje kolik pro každou proměnnou o kolik hodnot v tabulce se musíme posunout, abychom se dostali na další hodnotu této proměnné se zachováním hodnot všech následujících.

Příklad faktoru je v tabulce \ref{tab:stride}.
Doménu faktoru tvoří dvě binární proměnné $X$ a $Y$, jejich kardinalita je tedy 2.
Pro proměnnou $X$ je krok 2, pro proměnnou $Y$ je krok 1.

\begin{table}
\begin{center}
\begin{tabular}{|c|c|r|}
\hline
$X$ & $Y$ & Hodnota \\
\hline
\hline
0 & 0 & 0.2 \\
\hline
0 & 1 & 0.3 \\
\hline
1 & 0 & 0.1 \\
\hline
1 & 1 & 0.4 \\
\hline
\end{tabular}
\end{center}
\caption{Příklad faktoru s dvěma proměnnými $X$ a $Y$.}
\label{tab:stride}
\end{table}

\subsection{Operace s faktory}

Implementace diskrétního faktoru obsahuje všechny základní matematické operace, ale také speciální operace, které jsou využity specificky pro pravděpodobnostní rozložení, např. marginalizace anebo normalizace.
Operace jako násobení a marginalizace jsou používány při každém výpočtu zprávy a tedy je třeba je napsat tak, aby fungovaly co nejefektivněji.

Operace s faktory musí fungovat ve třech různých situacích, příklady uvedeme na násobení.
\begin{enumerate}
    \item Násobení faktoru s faktorem, oba se stejnou doménou,
    \item násobení dvou faktorů, které sdíli jen některé proměnné,
    \item násobení faktoru konstantou.
\end{enumerate}

Násobení faktoru konstantou je triviální, každá položka faktoru bude vynásobena konstantou.
Tato operace může být jednoduše vektorizována.

Při násobení dvou faktorů se stejnou doménou je třeba pronásobit prvky se stejným přiřazením proměnných.
Což znamená pronásobit hodnoty na stejných místech v poli.
Opět se tedy jedná o operaci, která je jednoduše vektorizovatelná.

\subsubsection{Operace s různými doménami}

Poslední možnost je, že se snažíme provést matematickou operaci s dvěma faktory, které ovšem nemají stejnou doménu.
Pak musí výsledkem být nový faktor, jehož doména je sjednocením domén vstupních faktorů.
Jednotlivé prvky nového faktoru jsou pak výsledkem aplikace operace na prvky ze vstupních faktorů, které sdílí ohodnocení společných proměnných. 
Příklad s násobením:
\begin{equation*}
\begin{array}{|c|c|r|}
    \hline
    \multicolumn{3}{|c|}{f_1} \\
    \hline
    X & Y & Hodnota \\
    \hline
    \hline
    0 & 0 & 0.2 \\
    \hline
    0 & 1 & 0.3 \\
    \hline
    1 & 0 & 0.1 \\
    \hline
    1 & 1 & 0.4 \\
    \hline
\end{array}
\times
\begin{array}{|c|c|r|}
    \hline
    \multicolumn{3}{|c|}{f_2} \\
    \hline
    Y & Z & Hodnota \\
    \hline
    \hline
    0 & 0 & 0.2 \\
    \hline
    0 & 1 & 0.2 \\
    \hline
    1 & 0 & 0.2 \\
    \hline
    1 & 1 & 0.4 \\
    \hline
\end{array}
=
\begin{array}{|c|c|c|r|}
    \hline
    \multicolumn{4}{|c|}{f_r} \\
    \hline
    X & Y & Z & Hodnota \\
    \hline
    \hline
    0 & 0 & 0 & 0.04 \\
    \hline
    0 & 0 & 1 & 0.04 \\
    \hline
    0 & 1 & 0 & 0.06 \\
    \hline
    0 & 1 & 1 & 0.12 \\
    \hline
    1 & 0 & 0 & 0.02 \\
    \hline
    1 & 0 & 1 & 0.02 \\
    \hline
    1 & 1 & 0 & 0.08 \\
    \hline
    1 & 1 & 1 & 0.16\\
    \hline
\end{array}
\end{equation*}

Výsledek násobení faktorů $f_1$ a $f_2$ je ve faktoru $f_r$.
Faktory sdílí pouze proměnnou $Y$, takže je potřeba pronásobit všechny přiřazení z $f_1$ se všemi přiřazeními z $f_2$, které mají stejnou hodnotu $Y$.
Příkladem je například přiřazení $(0, 1)$ s hodnotou $0.3$ vynásobené s hodnotou $(1, 1)$ s hodnotou $0.4$.
Výsledek je uložen ve faktoru $f_r$ s přiřazením $(0, 1, 1)$ a správnou hodnotou $0.3 \cdot 0.4 = 0.12$.

\subsection{Algoritmus pro operace s různými doménami}

Předvedli jsme možné případy operací s faktory a ukázali, že dva ze tří jsou triviální na implementaci.
Nyní představíme efektivní implementaci třetí možnosti, tedy aplikace operace na dva faktory s různými doménami (algoritmus \ref{alg:apop}).

Ze vstupních faktorů vytvoříme prázdný faktor pro výsledek.
Jeho doména je sjednocením domén vstupních faktorů.
Kardinalita proměnných zůstává stejná.
Krok jednotlivých proměnných je třeba přepočítat.
Spočítáme jej jako součin kardinalit proměnných, které následují po té aktuální.
Velikost pole pro všechny hodnoty je rovna součin všech kardinalit.

Následně přistoupíme k vyplňování tabulky.
Pro oba vstupní faktory si budeme udržovat index na pozici s ohodnocením proměnných, které odpovídá aktuálně vyplňovanému ohodnocení ve výsledném faktoru.
Po provedení operace tyto indexy aktualizujeme.

Pokud reprezentujeme ohodnocení proměnných jako číslo (kde každá cifra může mít jinou kardinalitu), pak se přesuneme k dalšímu ohodnocení v řadě tak, že zvýšíme nejméně signifikantní cifru (proměnnou) o jedna.
Může se stát, že jsme dosáhli kardinality dané proměnné a pak se musíme vrátit na ohodnocení 0 a aplikovat přesun na vyšší cifru.
Opakovanou aplikací přesunu můžeme upravit všechny proměnné, příkladem je přechod z ohodnocení $(0, 1, 1)$ na $(1, 0, 0)$, všechny proměnné binární.
Při každé úpravě proměnné také aktualizujeme indexy ve vstupních faktorech.

Pokud se přesunujeme na další hodnotu proměnné, tak stačí k indexu pro faktor přičíst krok upravené proměnné.
Pokud je třeba se vrátit na ohodnocení 0, pak od indexu pro vstupní faktor odečteme $(c_v - 1) \cdot s_v$, kde $c_v$ je kardinalita proměnné $v$ a $s_v$ je krok proměnné $v$.

\begin{algorithm}
\caption{Aplikace operace na faktory s různými doménami}
\label{alg:apop}
\begin{algorithmic}
\Function{Apply-Op}{$f_1$, $f_2$}
\EndFunction
\end{algorithmic}
\end{algorithm}
