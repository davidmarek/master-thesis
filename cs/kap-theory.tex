% !TEX root = prace.tex
\chapter{Teorie dialogových systémů}

\section{Dialogový systém}

Dialogový systém je počítačový systém, který umožňuje uživatelům používat
hlasových pokynů pro ovládání počítačů a získávání informací. Dialogové systémy
nejsou pouhým hlasovým ovládáním počítače. Obsahují strategii řízení dialogu a
navíc obsahují interní stav dialogu, díky kterému jsou schopny vést dialog s
uživatelem. Dialog pak umožňuje využít složitějších příkazů, pokládat dotazy
nad strukturovanými daty a upřesňovat dotaz.

Dialogové systémy stále mají spoustu problémů k překonání a pro praktické použití je třeba se uchýlit k několika předpokladům a zjednodušením.
Prvním zjednodušením je doménová specializace, v současné době není možné vytvořit dialogový systém, který by se dokázal s uživatelem bavit o libovolném tématu. 
Vždy je potřeba při vývoji dialogového systému vědět, jaké informace má systém poskytovat a o čem se může chtít uživatel bavit.
Další zjednodušení se týkají přímo dialogu. 
Předpokládá se, že dialog probíhá vždy mezi systémem a jedním uživatelem. Navíc se tito pravidelně střídají. Jedna obrátka dialogu je složená z jedné promluvy systému a jedné promluvy uživatele.

Příkladem dialogového systému může být systém pro nalezení spojení pomocí
městské dopravy. Příkladem příkazu od uživatele pak může být např. \uv{Chci jet z
Malostranského náměstí na Anděl}. Dialogový systém se nyní může rozhodnout,
zda-li mu zadané informace stačí pro nalezení spojení. V tomto případě systém
stále neví, kdy chce uživatel jet. Může předpokládat, že uživatel už na
zastávce stojí a tak tedy uživateli nalezne nejbližší spojení.

Důležitou vlastností dialogového systému je robustnost. Pokud budeme používat
dialogový systém v přirozeném prostředí, tak se musíme vyrovnat s tím, že často
nebude uživateli rozumět. Uživatel může chtít jet stejně jako v minulém
případě, ale nyní dialogový systém přeslechne zastávku Anděl. Pro nalezení
spojení už nemá dost informací, může si pořád domyslet, že uživatel chce jet
nyní, ale je mnohem více možností, kterým směrem se vůbec chce uživatel vydat.
Dialogový systém se proto uživatele zeptá na chybějící informace a zároveň může
i implicitně potvrdit, že rozuměl alespoň odchozí zastávku: \uv{Dobře, jedete ze
zastávky Malostranské náměstí, můžete říct na jakou zastávku chcete jet?}.

\subsection{Typy dialogových systémů}

Dialogové systémy můžeme dělit podle několika kritérií. Podle toho, která
strana dialog vede anebo třeba jakým způsobem funguje strategie dialogového
systému.

\subsubsection{Dělení podle iniciativy}

Nejjednodušší dialogové systémy fungují na stejném principu jako automatické
telefonní systémy. Systém je ten, kdo vede dialog, a uživatel pouze odpovídá na
dotazy nebo vybírá z nabídnutých možností. Tento systém je velmi jednoduchý na
implementaci, pokud nutíme uživatel, aby vždy pouze odpověděl na položený
dotaz, tak velmi omezíme jeho vyjadřovací možnosti a díky tomu bude mnohem
jednodušší rozpoznat co řekl. Například pokud se uživatele ptáme jestli musí
nalezené spojení obsahovat pouze bezbariérové dopravní prostředky, tak můžeme
očekávat pouze \uv{Ano} nebo \uv{Ne} a jejich synonyma. Tento přístup ovšem
nebude velmi populární mezi častými uživateli systému, kteří by rádi zrychlili
postup dialogem.

Další možností je tedy povolit uživateli sdělit informace i tehdy, kdy se ho
systém ptá na něco jiného. Systém se tedy stále ptá na informace, které mu
chybějí, ale může dialog začít obecnější otázkou: "Jaké spojení chcete nalézt?"
oproti "Z jaké zastávky chcete jet?". V prvním případě dává systém pokročilému
uživateli možnost popsat celou cestu a velmi rychle se tak dostat ke kýžené
informaci. Pro nové uživatele, ale může být takováto obecně položená otázka
matoucí a tak je většinou třeba přidat i příklad.

Posledním typem je takový, kde uživateli přenecháme iniciativu, příkladem může
být například systém \uv{How may I help you?} \cite{gorin1997may} od
společnosti AT\&T. Ten slouží pro klasifikaci požadavků od uživatele a jejich
přepojení k patřičnému operátorovi.

\subsubsection{Dělení podle strategie řízení}

Důležité rozdělení dialogových systémů je podle způsobu, jakým funguje jejich
strategie řízení. Systém většinou předpokládá, že se v dialogu střídá s
uživatelem. Vždy, když na systém přijde řada, tak se musí rozhodnout, zda-li už
má dost informací k zodpovězení dotazu uživatele, případně zda-li se musí
dotázat na další informace, nechat si nějakou informaci potvrdit od uživatele,
anebo nechat uživatele vybrat si z nabízených možností.

Systém může mít přístup do databáze s informacemi a musí se tedy umět
rozhodnout, zda-li je třeba v ní hledat. Pak musí samozřejmě na nalezená data
reagovat. Může se stát, že je informací, které odpovídají zadání uživatele
příliš velké množství a tak je třeba zpřísnit zadání, anebo naopak nebyla žádná
informace nalezena a tedy je potřeba uživatele o tomto informovat.

Strategii, podle které se systém bude v těchto případech řídit lze
implementovat několika způsoby. Přímým řešením je napsat ji ručně. Pak se
rozhodnutí systému skládá z procházení série podmínek až do nalezení podmínky
odpovídající aktuálnímu stavu a následné provedení akce specifikované touto
podmínkou. Tento způsob může fungovat pro jednoduché systémy a pokud je vhodně
napsaný, tak může fungovat velmi dobře. Dost často ale tento způsob selhává,
pokud je dialogový systém příliš složitý, když se stávají podmínky
neudržitelné, anebo v případě, kdy se nemůžeme spolehnout na rozpoznanou řeč.

Další možností jak vytvořit strategii řízení je použít učení s učitelem.
Předpokládáme, že máme databázi hovorů mezi uživatelem a živým operátorem, z
kterých se dialogový systém může naučit, jak v dialogu postupovat. Tento
přístup se zdá jako dobrý nápad, ovšem naučíme se pouze napodobit operátora,
což nemusí být ten nejlepší způsob jak vést dialog. Navíc je nepřeberné
množství směrů, kterým se může dialog vyvíjet a jakmile se v jedné akci
odchýlíme, tak dostáváme úplně jiný dialog a tedy je velmi těžké z pozorovaných
dialogů zobecňovat. Proto se tento způsob téměř nevyužívá, existují ovšem
pokusy jak jej skloubit s následující metodou učení \cite{levin2000stochastic}.

Nejpoužívanější metodou v současnosti jak učit strategii dialogového systému
je použít zpětnovazební učení \cite{singh1999reinforcement}, \cite{walker2011application}. 
V takovém případě nám stačí pouze zhodnocení výsledku dialogu, které můžeme získat požádáním uživatelů dialogového systému, aby na konci dialogu oznámkovali, jak byli s výsledkem spokojeni. 
Při učení se nesnažíme napodobit existujícího operátora, ale získat co nejlepší hodnocení od uživatelů.
Díky tomu se může dialogový systém naučit pracovat dokonce ještě lépe než operátor.

\section{Součásti dialogového systému}

Dialogový systém se skládá z několika částí, každá se specializuje na jiný úkol a dohromady tvoří spolupracující systém.
Na vstupu je vždy zvukový záznam, o převedení do textu se stará systém rozpoznávání řeči (ASR).
Z textu je potřeba získat sémantické informace pomocí systému porozumění mluvené řeči (SLU).
Nad sémanticky anotovanými informacemi už může pracovat dialogový manager (DM), který rozhoduje o další akci.
Výstupem dialogového manageru jsou informace, které se mají předat uživateli.
O jejich převedení do textu se stará systém generování přirozené řeči (NLG).
Do zvukového záznamu převede text syntetizér řeči (TTS).

\subsection{Systém rozpoznávání řeči (ASR)}

Systém rozpoznávání řeči slouží k převedení mluveného projevu do textové podoby. 
Po získání textové podoby je teprve možné se zabývat významem textu.
Aktuálně nejlepší systémy jsou založené na pravděpodobnostním modelu a využívají skryté Markovské modely (HMM). 
Nejznámější je systém HTK \cite{young2002htk}, dalším otevřeným systémem je např. Kaldi \cite{Povey_ASRU2011}.
Existuje i celá řada komerčního software od firem jako IBM nebo Nuance.

Úspěšnost systému rozpoznání řeči je závislá na obtížnosti úlohy a na počtu trénovacích dat, pocházejících ze stejné domény.
Pro obecnou doménu se problém stává mnohem těžší.
Příkladem obecného rozpoznávače je Google Voice Search.

Systém rozpoznávání řeči může produkovat více hypotéz pro jeden vstup.
Často existuje pro jeden zvukový záznam více možných slovních sekvencí, z kterých by mohl pocházet.
Jejich reprezentace může být seznamem možných hypotéz s věrohodností pro každou hypotézu.
Věrohodnosti jsou skóre přiřazené hypotézám, které určují jakou důvěru má systém rozpoznávání řeči ve správnost dané hypotézy.
V ideálním případě je věrohodnost ekvivalentní aposteriorní pravděpodobnosti sekvence slov, dáno vstupní zvuk.
Ovšem ne všechny rozpoznávače pracují na pravděpodobnostním principu a pak není možné od nich požadovat skutečné pravděpodobnosti.

Další možností jak reprezentovat výstup je použití konfůzní sítě \cite{bertoldi2005new}. 
Konfůzní síť je vážený orientovaný graf obsahující startovní a konečný vrchol a hrany má označené slovy. 
Každá cesta ze startovního do konečného vrcholu vede přes všechny ostatní vrcholy. 
Váhy hran jsou pravděpodobnosti slova přiřazeného dané hraně. 
Hrany mohou obsahovat i prázdné slovo $\epsilon$.
Pravděpodobnost sekvence slov je součinem vah po cestě ze startovního do konečného uzlu.
Výhodou konfůzní sítě je, že umožňuje v komprimované podobě uložit mnohem více hypotéz.

\subsection{Porozumění mluvené řeči (SLU)}

Systém SLU převádí text do dialogových aktů. 
Dialogový akt (DA) je reprezentace promluvy, skládá se z jedné nebo více položek dialogového aktu (DAI), které jsou spojené v konjunkci.
Každá DAI se skládá z typu, názvu slotu a jeho hodnoty. 
Příklad DA z dialogového systému pro hledání restaurací:

\begin{center}
{\tt hello()\&inform(food="chinese")}.
\end{center}

Zde se DA skládá ze dvou položek, první položka má pouze typ {\em hello}, značící pozdrav. 
Druhá položka má typ {\em inform}, tzn. uživatel nás informuje o svém požadavku.
Název slotu je {\em food} a hodnota je \uv{chinese}, tedy uživatel nám říká, že hledá restauraci, kde servírují čínské jídlo.

Typů může být libovolné množství, ale existuje několik základních, jejichž použití je ustálené. 
\begin{itemize}
\item {\em inform} --- sdělujeme informaci, doplňujeme hodnotu do slotu,
\item {\em request} --- požadujeme od protějšku doplnění hodnoty pro dotazovaný slot,
\item {\em confirm} --- chceme potvrdit hodnotu slotu, potvrzení může být implicitní, anebo explicitní. 
	Při explicitním potvrzení očekáváme odpověď buď \uv{Ano} nebo \uv{Ne},
	U implicitního, pokud se nám nedostane odpovědi předpokládáme, že protějšek souhlasí,
\item {\em select} --- žádáme protějšek, aby zvolil z nabízených možností.
\end{itemize}

\subsection{Dialogový manager (DM)}

Dialogový manager tvoří mozek dialogového systému, jeho vstupem jsou dialogové akty z SLU, výstupem opět dialogové akty, pro NLG.
DM se dělí na dvě části. 
První část se stará o udržování dialogového stavu, tzn. informace, které uživatel systému poskytl, historii dialogu, které informace už byly poskytnuty anebo např. potvrzeny či zamítnuty uživatelem.
Dále obsahuje dialogovou strategii, která určuje příští akci v závislosti na dialogovém stavu.

\subsubsection{Reprezentace stavu dialogu}

Markovský rozhodovací proces (MDP) \cite{puterman2009markov} je diskrétní, stochastický a kontrolovaný proces. 
V každém časovém okamžiku se systém nachází v nějakém stavu $s$. 
Uživatel provede akci $a$, dostupnou ve stavu $s$, a ta jej přesune náhodně do nového stavu $s^\prime$ a navíc dostane odměnu $r(s, s^\prime, a)$.
Pravděpodobnost přechodu ze stavu $s$ do stavu $s^\prime$ je dána přechodovou funkcí $p(s^\prime \mid s, a)$.

Zobecněním MDP je částečně pozorovatelný Markovský rozhodovací proces (POMDP). 
Částečně pozorovatelný proto, že na rozdíl od MDP nevíme v jakém stavu se systém nachází.
Naše představa o stavu je dána pouze pravděpodobnostním rozložením přes všechny možné stavy, tzv. belief $b(s)$.
Výsledek provedení akce tedy nezávisí pouze na přechodové pravděpodobnosti, ale také na belief stavu. 
Pro aktualizaci stavu je potřeba vysčítat přes všechny možné stavy
\begin{equation}
b(s^\prime) = \sum_s p(s^\prime \mid s, a) b(s).
\end{equation}

Dialog je tedy popsán pomocí POMDP \cite{williams2007partially}, protože musíme zakomponovat naši neznalost cílů uživatele a problémy s rozpoznáváním. 
Pro reprezentaci stavu dialogu existuje několik různých přístupů.

\section{Grafické modely}

