% !TEX root = prace.tex
\chapter{Teorie}

\section{Dialogový systém}

Dialogový systém je počítačový systém, který umožňuje uživatelům používat
hlasových pokynů pro ovládání počítačů a získávání informací. Dialogové systémy
nejsou pouhým hlasovým ovládáním počítače. Obsahují strategii řízení dialogu a
navíc obsahují interní stav dialogu, díky kterému jsou schopny vést dialog s
uživatelem. Dialog pak umožňuje využít složitějších příkazů, pokládat dotazy
nad strukturovanými daty a upřesňovat dotaz.

Příkladem dialogového systému může být systém pro nalezení spojení pomocí
městské dopravy. Příkladem příkazu od uživatele pak může být např. \uv{Chci jet z
Malostranského náměstí na Anděl}. Dialogový systém se nyní může rozhodnout,
zda-li mu zadané informace stačí pro nalezení spojení. V tomto případě systém
stále neví, kdy chce uživatel jet. Může předpokládat, že uživatel už na
zastávce stojí a tak tedy uživateli nalezne nejbližší spojení.

Důležitou vlastností dialogového systému je robustnost. Pokud budeme používat
dialogový systém v přirozeném prostředí, tak se musíme vyrovnat s tím, že často
nebude uživateli rozumět. Uživatel může chtít jet stejně jako v minulém
případě, ale nyní dialogový systém přeslechne zastávku Anděl. Pro nalezení
spojení už nemá dost informací, může si pořád domyslet, že uživatel chce jet
nyní, ale je mnohem více možností, kterým směrem se vůbec chce uživatel vydat.
Dialogový systém se proto uživatele zeptá na chybějící informace a zároveň může
i implicitně potvrdit, že rozuměl alespoň odchozí zastávku: \uv{Dobře, jedete ze
zastávky Malostranské náměstí, můžete říct na jakou zastávku chcete jet?}.

\subsection{Typy dialogových systémů}

Dialogové systémy můžeme dělit podle několika kritérií. Podle toho, která
strana dialog vede anebo třeba jakým způsobem funguje strategie dialogového
systému.

\subsubsection{Dělení podle iniciativy}

Nejjednodušší dialogové systémy fungují na stejném principu jako automatické
telefonní systémy. Systém je ten, kdo vede dialog, a uživatel pouze odpovídá na
dotazy nebo vybírá z nabídnutých možností. Tento systém je velmi jednoduchý na
implementaci, pokud nutíme uživatel, aby vždy pouze odpověděl na položený
dotaz, tak velmi omezíme jeho vyjadřovací možnosti a díky tomu bude mnohem
jednodušší rozpoznat co řekl. Například pokud se uživatele ptáme jestli musí
nalezené spojení obsahovat pouze bezbariérové dopravní prostředky, tak můžeme
očekávat pouze \uv{Ano} nebo \uv{Ne} a jejich synonyma. Tento přístup ovšem
nebude velmi populární mezi častými uživteli systému, kteří by rádi zrychlili
postup dialogem.

Další možností je tedy povolit uživateli sdělit informace i tehdy, kdy se ho
systém ptá na něco jiného. Systém se tedy stále ptá na informace, které mu
chybějí, ale může dialog začít obecnější otázkou: "Jaké spojení chcete nalézt?"
oproti "Z jaké zastávky chcete jet?". V prvním případě dává systém pokročilému
uživateli možnost popsat celou cestu a velmi rychle se tak dostat ke kýžené
informaci. Pro nové uživatele, ale může být takováto obecně položená otázka
matoucí a tak je většinou třeba přidat i příklad.

Posledním typem je takový, kde uživateli přenecháme iniciativu, příkladem může
být například systém \uv{How may I help you?} \todo{Odkaz na článek o HMIHY} od
společnosti AT\&T. Ten slouží pro klasifikaci požadavků od uživatele a jejich
přepojení k patřičnému operátorovi.

\subsubsection{Dělení podle strategie řízení}

Důležité rozdělení dialogových systémů je podle způsobu, jakým funguje jejich
strategie řízení. Systém většinou předpokládá, že se v dialogu střídá s
uživatelem. Vždy, když na systém přijde řada, tak se musí rozhodnout, zda-li už
má dost informací k zodpovězení dotazu uživatele, případně zda-li se musí
dotázat na další informace, nechat si nějakou informaci potvrdit od uživatele,
anebo nechat uživatele vybrat si z nabízených možností.

Systém může mít přístup do databáze s informacemi a musí se tedy umět
rozhodnout, zda-li je třeba v ní hledat. Pak musí samozřejmě na nalezená data
reagovat. Může se stát, že je informací, které odpovídají zadání uživatele
příliš velké množství a tak je třeba zpřísnit zadání, anebo naopak nebyla žádná
informace nalezena a tedy je potřeba uživatele o tomto informovat.

Strategii, podle které se systém bude v těchto případech řídit lze
implementovat několika způsoby. Přímým řešením je napsat ji ručně. Pak se
rozhodnutí systému skládá z procházení série podmínek až do nalezení podmínky
odpovídající aktuálnímu stavu a následné provedení akce specifikované touto
podmínkou. Tento způsob může fungovat pro jednoduché systémy a pokud je vhodně
napsaný, tak může fungovat velmi dobře. Dost čast ale tento způsob selhává
pokud je dialogový systém příliš složitý, když se stávají podmínky
neudržitelné, anebo v případě, kdy se nemůžeme spolehnout na rozpoznanou řeč.

Další možností jak vytvořit strategii řízení je použít učení s učitelem.
Předpokládáme, že máme databázi hovorů mezi uživatelem a živým operátorem, z
kterých se dialogový systém může naučit, jak v dialogu postupovat. Tento
přístup se zdá jako dobrý nápad, ovšem naučíme se pouze napodobit operátora,
což nemusí být ten nejlepší způsob jak vést dialog. Navíc je nepřeberné
množství směrů, kterým se může dialog vyvíjet a jakmile se v jedné akci
odchýlíme, tak dostáváme úplně jiný dialog a tedy je velmi těžké z pozorovaných
dialogů zobecňovat. Proto se tento způsob téměř nevyužívá, existují ovšem
pokusy jak jej skloubit s následující metodou učení \todo{citovat pieraccini
2000}.
