% !TEX root = prace.tex
\chapter{Učení parametrů}

\section{Grafický model}

Máme vybraný faktor $f$, tento faktor je spojený s několika proměnnými
$\vec{x} = (x_0, x_1, \dots, x_{N_x})$
a množinami parametrů
$\vec{\Theta} = (\vec{\theta}_1, \dots, \vec{\theta}_{N_\theta})$.
Tento faktor reprezentuje podmíněnou pravděpodobnost:
$$f(\vec{x}, \Theta) = p(x_0 | x_1, \dots, x_{N_x}; \Theta)$$
Rodičovské proměnné $x_1, \dots, x_{N_x}$ označujeme jako $\vec{x^\prime}$.
Vektor $\vec{x^\prime}$ určuje, která množina parametrů bude použita.
Protože množiny parametrů jsou číslovány $1, \dots, N_\theta$ a rodičovské
proměnné $1, \dots, N_x$, musí být pro vybrání správné množiny parametrů
použito mapování $\rho(\vec{x^\prime})$.
Faktor pak může být zapsán zkráceně:
$$f(\vec{x}, \Theta) = p(x_0 | x_1, \dots, x_{N_x}; \Theta) =
\theta_{\rho(x^\prime), x_0}$$

\section{Výpočet marginálních pravděpodobností}

Pro výpočet sdružené pravděpodobnosti používáme plně faktorizovanou distribuci.
Pro každou proměnnou anebo množinu parametrů je její marginální pravděpodobnost
rovna součinu zpráv přicházejících z faktorů, které jsou s danou proměnnou
nebo množinu parametrů propojeny.
Pro daný faktor je cavity distribuce $q^\backslash(x_i)$, popř.
$q^\backslash(\vec{\theta}_i)$ rovna součinu zpráv ze všech ostatních faktorů do $x_i$, popř $\vec\theta_i$.
Aproximovaná marginální pravděpodobnost proměnné je pak součinem cavity
distribuce a zprávy z faktoru:
$$q(x_i) = q^\backslash(x_i) m_{f \rightarrow x_i}(x_i)$$
$$q(\vec{\theta}_i) = q^\backslash(\vec{\theta}_i) m_{f\rightarrow \theta}(\vec{\theta}_i)$$

Cavity distribuce je právě zpráva z proměnné, popř. množiny parametrů do faktoru.

$$m_{x_i \rightarrow f} = q^\backslash (x_i)$$
$$m_{\vec\theta_i \rightarrow f} = q^\backslash (\vec\theta_i)$$

\subsection{Marginální pravděpodobnost proměnných}

Pokud chceme aktualizovat hodnotu naší aproximace marginální pravděpodobnosti,
tak je třeba minimalizovat její vzdálenost od skutečné marginální
pravděpodobnosti:
\begin{align}
p^*(\tilde{x}_j) &=
\sum_{\vec{x}: x_j = \tilde{x}_j}
	\int_{\vec{\Theta}}
    		\prod_i 
			q^\backslash(x_i)
		\prod_l 
			q^\backslash(\vec{\theta}_l)
		f(\vec{x};
    		  \vec{\Theta})
\label{eq:1} 
\\
&=
\sum_{\vec{x}: x_j = \tilde{x}_j} 
	\prod_i 
		q^\backslash(x_i)
    \int_{\vec{\theta_{\rho(\vec{x^\prime})}}}
	    q^\backslash(\vec{\theta}_{\rho(\vec{x^\prime})})
    \theta_{\rho(\vec{x^\prime}), x_0} \label{eq:2} 
\\
&= 
\sum_{\vec{x}: x_j = \tilde{x}_j} 
	\prod_i 
		q^\backslash(x_i)
    		\mathbb{E}_{q^\backslash} 
			(\theta_{\rho(\vec{x^\prime}), x_0}) 
\label{eq:3}
\\
&= 
\sum_{\vec{x}: x_j = \tilde{x}_j} 
	\prod_i 
		m_{x_i \rightarrow f}(x_i)
    		\mathbb{E}_{q^\backslash} 
			(\theta_{\rho(\vec{x^\prime}), x_0}) 
\end{align}

Rovnost~(\ref{eq:1}) vychází z definice výpočtu marginální pravděpodobnosti ze
sdružené pravděpodobnosti.
V~(\ref{eq:2}) byla použita definice faktoru, z integrálu byly vytaženy členy,
které neobsahují $\Theta$ a nakonec bylo využito toho, že pro
množiny parametrů, které nejsou spojeny s faktorem $f$, je jejich jejich cavity
distribuce rovná marginální distribuci a tedy $\int_{\theta_i} q(\theta_i) =
1$. V~(\ref{eq:3}) byla použita definice očekávané hodnoty.

Marginální pravděpodobnost proměnné $x_i$ tedy je
\begin{equation}
p^*(\tilde x) =
\sum_{\vec{x}: x_j = \tilde{x}_j} 
	\prod_i 
		m_{x_i \rightarrow f}(x_i)
    		\mathbb{E}_{q^\backslash} 
			(\theta_{\rho(\vec{x^\prime}), x_0}) 
\end{equation}

Tady docházíme k výsledku, který je velmi podobný výpočtu marginální
pravděpodobnosti v Loopy Belief Propagation algoritmu.

Zprávu z faktoru $f$ do vrcholu $x_j$ pak získáme vydělením zprávy z $x_j$ z
marginální pravděpodobnosti.

\begin{equation}
m_{f \rightarrow x_j}(x_j) =
    \sum_{\vec{x}: x_j = \tilde{x}_j}
        \prod_{i \ne j}
            m_{x_i \rightarrow f}(x_i)
            \mathbb{E}_{q^\backslash}
                (\theta_{\rho(\vec{x^\prime}), x_0})
\label{eq:msgfromftox}
\end{equation}

\subsection{Marginální pravděpodobnost parametrů}

Pro množiny parametrů se jejich marginální pravděpodobnost spočítá podobně jako
pro proměnné.
\begin{align}
p^*(\tilde{\vec{\theta}}_j) & = \sum_{\vec{x}} \int_{\vec{\Theta}:
    \vec{\theta}_j = \tilde{\vec{\theta}}_j} \prod_i q^\backslash(x_i) \prod_l
    q^\backslash(\vec{\theta}_l) f(\vec{x}; \vec{\Theta}) \label{eq:ep:theta_1}
\\
& = \sum_{l \ne j} \sum_{\vec{x}: \rho(\vec{x^\prime}) = l} \prod_i
    q^\backslash(x_i) \int_{\vec{\Theta}: \vec{\theta}_j =
    \tilde{\vec{\theta}}_j} \prod_k q^\backslash(\vec{\theta}_k) \theta_{l,
    x_0} + \label{eq:ep:theta_2}
\\
&   + \sum_{\vec{x}: \rho(\vec{x^\prime}) = j} \prod_i q^\backslash(x_i)
    \int_{\vec{\Theta}: \vec{\theta}_j = \tilde{\vec{\theta}}_j} \prod_k
    q^\backslash(\vec{\theta}_k) \tilde{\theta}_{j, x_0}
\nonumber
\\
& = \left[ \sum_{l \ne j} \sum_{\vec{x}: \rho(\vec{x^\prime}) = l} \prod_i
    q^\backslash(x_i) \mathbb{E}_{q^\backslash(\vec{\theta}_l)} (\theta_{l,
    x_0}) \right] q^\backslash(\tilde{\vec{\theta}}_j) + \label{eq:ep:theta_3}
\\
&   + \sum_{\vec{x}: \rho(\vec{x^\prime}) = j} \prod_i q^\backslash(x_i)
    \tilde{\theta}_{j,x_0} q^\backslash(\tilde{\vec{\theta}}_j)
\label{eq:ep:theta_4}
\nonumber
\\
& = w_0 q^\backslash(\tilde{\vec{\theta}}_j) + \sum_k w_k
    \tilde{\theta}_{j,k} q^\backslash(\tilde{\vec{\theta}}_j),
\\
& = w_0 m_{\tilde{\vec\theta}_j \rightarrow f}(\tilde{\vec{\theta}}_j) + \sum_k w_k
    \tilde{\theta}_{j,k} m_{\tilde{\vec\theta}_j \rightarrow f}(\tilde{\vec{\theta}}_j),
\end{align}

kde

\begin{align}
w_0 &=
	\sum_{l \ne j} 
		\sum_{\vec{x}: \rho(\vec{x^\prime}) = l} 
			\prod_i
 				m_{x_i \rightarrow f}(x_i)
				\mathbb{E}_{q^\backslash(\vec{\theta}_l)} (
					\theta_{l,
					x_0}) 
\\
w_k &=
	\sum_{\vec{x}: \rho(\vec{x^\prime}) = j, x_0 = k} 
		\prod_i
			m_{x_i \rightarrow f}(x_i)		
\end{align}

Opět vycházíme z výpočtu marginální pravděpodobnosti ze sdružené
pravděpodobnosti. V rovnici \eqref{eq:ep:theta_2} jsme rozdělili sumu přes
$\vec{x}$ na ty, pro které se ve faktoru použije množina parametrů
$\tilde{\vec{\theta}}_j$ a na ty ostatní. Také jsme z integrálu vytknuli součin
cavity distribucí pro proměnné. V dalším kroku \eqref{eq:ep:theta_3} jsme opět
použili toho, že integrál přes $\vec{\Theta}$ je ve skutečnosti několik
integrálů přes jednotlivé množiny parametrů. A tedy je můžeme vložit mezi
jednotlivé členy produktu cavity distribucí pro množiny parametrů. Ve výsledku
získáme $q^\backslash(\tilde{\vec{\theta}}_j) \int_{\vec{\theta}_l}
q^\backslash(\vec{\theta}_l) \theta_{l, x_0}$ a pak zbylé členy, které zmizí.

Docházíme k vyjádření skutečné marginální pravděpodobnosti, ve které není třeba
integrovat přes všechny množiny parametrů, ale stačí jen očekávaná hodnota
těchto parametrů.

\section{Aproximace marginálních pravděpodobností}

Stále tu ovšem zůstává problém, že spočítat aproximující distribuci
$q(\vec{\theta}_j)$ může být příliš složité, protože skutečná marginální
distribuce je směs několika distribucí a ta nemusí být v obecném případě
vyjádřitelná. Je tedy třeba model dále aproximovat. Pro zjednodušení výpočtu
jsou zprávy z faktoru do množiny parametrů, $m_{f \rightarrow
\vec{\theta}_i}(\vec{\theta}_i)$, ve tvaru Dirichletovského rozdělení s
parametry $\vec\alpha_{f \rightarrow \vec\theta_i}$:
\begin{equation}
m_{f \rightarrow \vec{\theta}_i}(\vec{\theta}_i) =
	Dir(\vec{\theta}_i; \vec\alpha_{f \rightarrow \vec\theta_i}) =
        \frac{\Gamma (\sum_j \vec\alpha_{f \rightarrow \vec\theta_i, j})}
             {\prod_j \Gamma(\vec\alpha_{f \rightarrow \vec\theta_i, j})}
        \prod_j \theta_{i,j}^{\vec\alpha_{f \rightarrow \vec\theta_i, j} - 1}
\end{equation}
kde $\Gamma$ je Gamma funkce (zobecnění faktoriálu):
\begin{equation}
    \Gamma(z) = \int_0^\infty \! t^{z-1} \exp(-t) \mathrm{d}t
\end{equation}

Dirichletovské rozdělení bylo zvoleno, protože má důležité vlastnosti pro
součin, které budou využity dále pro výpočet cavity distribuce a celkové
aproximace. Pokud označíme aproximované faktory indexem $\beta$ a každý bude
mít vlastní parametry $\vec\alpha_{f_\beta \rightarrow \vec\theta_i}$, tak výsledná aproximace bude
tvaru:
\begin{align}
q(\vec{\theta}_i) &\propto
    \prod_\beta
    m_{f_\beta \rightarrow \vec{\theta}_i}(\vec{\theta}_i)
\\
&\propto
    \prod_\beta
        \prod_j
            \theta_{i,j}^{\vec\alpha_{f_\beta \rightarrow \vec\theta_i, j} - 1}
\\
&\propto
    Dir(\vec{\theta}_i;
        \sum_\beta \vec\alpha_{f_\beta \rightarrow \vec\theta_i} - (|\beta| - 1) \vec{1})
\\
&=
    Dir(\vec{\theta}_i;
        \vec{\alpha}_i)
\label{eq:ep:aprx_1}
\end{align}
kde $\vec{\alpha}_i = \sum_\beta \vec\alpha_{f_\beta \rightarrow \vec\theta_i} - (|\beta| - 1)
\vec{1})$.

Při aktualizaci faktoru $\tilde\beta$ tedy cavity distribuce bude:
\begin{align}
q^{\backslash \tilde\beta} (\vec{\theta_i})
&\propto
    \prod_{\beta \ne \tilde\beta}
        m_{f_\beta \rightarrow \vec{\theta}_i}(\vec{\theta}_i)
\\
& \propto
Dir(\vec{\theta}_i;
    \vec{\alpha}_i - \vec\alpha_{f_\beta \rightarrow \vec\theta_i} + \vec{1})
\end{align}

Naším cílem je nalézt parametry $\vec{\alpha}^*$ aproximované marginální
pravděpodobnosti \eqref{eq:ep:aprx_1}, které minimalizují vzdálenost od
skutečné marginální pravděpodobnosti \eqref{eq:ep:theta_4}. Pro měření
vzdálenosti mezi dvěma pravděpodobnostními rozloženími se používá
Kullback-Leiblerova divergence:
\begin{equation}
KL(p \| q) =
\int_{-\infty}^{\infty}
    p(x) \log\left(\frac{p(x)}{q(x)}\right) \mathrm{d}x
\end{equation}
Pro nalezení minima použijeme algoritmus Expectation Propagation a budeme tedy
minimalizovat $KL(p^*\| q)$.

Pokud se podíváme na skutečnou marginální pravděpodobnost
$p^*(\vec{\theta}_i)$, zjistíme, že můžeme některé její členy upravit.
Využijeme také vlastnosti gamma funkce $\Gamma(x) = (x-1) \Gamma(x-1)$.

\begin{align}
w_j \theta_j Dir(\vec{\theta}; \vec{\alpha}) &\propto
    w_j \theta_j \frac{\Gamma (\sum_i \alpha_{i})}{\prod_i \Gamma(\alpha_i)}
    \prod_i \theta_{i}^{\alpha_{i} - 1}
\label{eq:4}
\\
&\propto w_j \frac{\Gamma (\sum_i \alpha_{i})} {\prod_i
    \Gamma(\alpha_{i})} \theta_j^{\alpha_j} \prod_{i \ne j}
    \theta_{i}^{\alpha_{i} - 1} \\
&\propto w_j
    \frac{\Gamma (\sum_i \alpha_{i})}
         {\prod_i \Gamma(\alpha_{i})}
    \frac{\Gamma(\alpha_{j} + 1) \prod_{i \ne j} \Gamma(\alpha_i)}
         {\Gamma (1 + \sum_i \alpha_i)}
    Dir(\vec\theta; \vec{\alpha} + \vec{\delta}_j) \\
&\propto w_j
    \frac{\Gamma (\sum_i \alpha_{i})}
         {\prod_i \Gamma(\alpha_{i})}
    \frac{\alpha_j \Gamma(\alpha_j) \prod_{i \ne j} \Gamma(\alpha_i)}
         {(\sum_i \alpha_i) \Gamma (\sum_i \alpha_i)}
    Dir(\vec\theta; \vec{\alpha} + \vec{\delta}_j) \\
&\propto w_j
    \frac{\alpha_j}
         {\sum_i \alpha_i}
    Dir(\vec\theta; \vec{\alpha} + \vec{\delta}_j) \\
\end{align}

Díky této úpravě lze $p^*$ vyjádřit jako směs Dirichletovských rozdělení.
\begin{equation}
p^*(\vec{\theta}) =
    w_0^* Dir(\vec{\theta}; \vec{\alpha}) +
    \sum_j w^*_j
        Dir(\vec{\theta}; \vec{\alpha} + \vec{\delta}_j)
\label{eq:marginaltheta}
\end{equation}
kde
\begin{align}
    w^*_0 &\propto w_0 \\
    w^*_j &\propto w_j \frac{\alpha_j}{\sum_i \alpha_i} \\
    \sum_{i=0}^k w_i^* &= 1
\end{align}

Pro minimalizaci KL divergence mezi dvěma rozděleními z exponenciální rozdělení
stačí, pokud se budou rovnat jejich postačující statistiky. Dokážeme jednoduše
spočítat první dva momenty Dirichletovského rozdělení a tedy použijeme
aproximaci a budeme počítat pouze s nimi a zbylé momenty zanedbáme.
Je tedy třeba nalézt střední hodnotu a rozptyl proměnných z $p^*(\vec\theta)$.

\begin{align}
\mathbb{E}_{p^*}[\vec\theta] &= \int \vec\theta p^*(\vec\theta) ~ \mathrm{d}\vec\theta
\\
&= \int \vec\theta (
	w_0^* Dir(\vec{\theta}; \vec{\alpha}) +
	\sum_j w^*_j
        	Dir(\vec{\theta}; \vec{\alpha} + \vec{\delta}_j)
    ) ~ \mathrm{d}\vec\theta
\\
&= w_0^* \int \vec\theta Dir(\vec{\theta}; \vec{\alpha}) ~ d\vec\theta +
	\sum_j w_j^* \int \vec\theta Dir(\vec{\theta}; \vec{\alpha} + \vec{\delta}_j)
    ~ \mathrm{d}\vec\theta
\\
&= w_0^* \mathbb{E}_{Dir(\vec\alpha)}[\vec\theta] +
	\sum_j w_j^* \mathbb{E}_{Dir(\vec\alpha + \vec{\delta}_j)}[\vec\theta]
\end{align}
Střední hodnotu proměnných $\vec\theta$ podle rozdělení $p^*$ lze tedy spočítat
jako vážený součet středních hodnot $\vec\theta$ podle jednotlivých
Dirichletovských distribucí, z kterých se $p^*$ skládá.
Střední hodnota proměnné $X_i$ podle Dirichletovského rozdělení je

První moment tedy máme spočítáný, pro výpočet rozptylu můžeme využít přímo
definici:
\begin{equation}
	Var[X] = \mathbb{E}[X^2] - \mathbb{E}[X]^2
\end{equation}

Chybí nám tedy ještě výpočet střední hodnoty druhé mocniny proměnné
$\vec\theta$ podle $p^*$. Můžeme ji vyjádřit z definice střední hodnoty.

\begin{align}
\mathbb{E}_{p^*}[\vec\theta^2] &= \int \vec\theta^2 p^*(\vec\theta) ~ d\vec\theta
\\
&= w_0^* \int \vec\theta^2 Dir(\vec{\theta}; \vec{\alpha}) ~ d\vec\theta +
	\sum_j w_j^* \int \vec\theta^2 Dir(\vec{\theta}; \vec{\alpha} + \vec{\delta}_j)
	 ~ d\vec\theta
\\
&= w_0^* \mathbb{E}_{Dir(\vec\alpha)}[\vec\theta^2] +
	\sum_j w_j^* \mathbb{E}_{Dir(\vec\alpha + \vec{\delta}_j)}[\vec\theta^2]
\end{align}

Opět získáváme vážený součet středních hodnot podle Dirichletovských rozdělení.
Střední hodnotu druhé mocniny proměnné podle Dirichletovského rozdělení
lze opět jednoduše odvodit z definice.
\begin{align}
\mathbb{E}_{Dir(\vec\alpha)}[x_i^2] &=
	\int x_i^2 Dir(\vec x; \vec\alpha) d\vec x
\\
&=
	\int x_i^2 \frac{\Gamma(\alpha_0)}
			   {\prod_{j=1}^N \Gamma(\alpha_j)}
		\prod_{j=1}^N x_j^{\alpha_j - 1} d\vec x
\end{align}

Nyní jsme v podobné situaci jako v~(\ref{eq:4}). Budeme postupovat stejně,
vyjádříme nové Dirichletovské rozdělení.

\begin{align}
\mathbb{E}_{Dir(\vec\alpha)}[x_i^2] &=
\int \frac{\Gamma(\alpha_0 + 2) \alpha_i (\alpha_i + 1)}
		{\alpha_0 (\alpha_0 + 1) \Gamma(\alpha_i + 2) \prod_{j \ne i} \Gamma(\alpha_j)}
    x_i^{\alpha_i + 1} \prod_{j \ne i} x_j^{\alpha_j - 1}
\mathrm{d}\vec x
\\
&= \frac{\alpha_i (\alpha_i + 1)}
		{\alpha_0 (\alpha_0 + 1)}
	\int \frac{\Gamma(\beta_0)}
		{\prod_i \Gamma(\beta_i)}
        \prod_i x_i^{\beta_i - 1} 
    \mathrm{d}\vec x
\\
&= \frac{\alpha_i (\alpha_i + 1)}
		{\alpha_0 (\alpha_0 + 1)}
	\int Dir(\vec x; \vec\beta)
    \mathrm{d}\vec x
\\
&= \frac{\alpha_i (\alpha_i + 1)}
		{\alpha_0 (\alpha_0 + 1)}
\end{align}

Vyjádřili jsme $\Gamma(\alpha_0)$ a $\Gamma(\alpha_i)$ s pomocí $\Gamma(\alpha_0 + 2)$ a $\Gamma(\alpha_i + 2)$
\begin{align}
\Gamma(\alpha_0) &= \frac{\Gamma(\alpha_0 + 2)}
				{\alpha_0(\alpha_0 + 1)}
\\
\Gamma(\alpha_i) &= \frac{\Gamma(\alpha_i + 2)}
				{\alpha_i(\alpha_i + 1)}
\end{align}

Následně jsme vytvořili nové parametry $\vec\beta$:
\begin{align}
\beta_i &= \alpha_i + 2 \\
\beta_{j \ne i} &= \alpha_j \\
\beta_0 &= \sum_i \beta_i
\end{align}

Nyní tedy dokážeme spočítat $\mathbb{E}_{p^*}[\vec\theta]$ a $\mathbb{E}_{p^*}[\vec\theta^2]$.
Parametry aproximovaného rozdělení nalezneme následovně
\begin{align}
\frac{\mathbb{E}[X_1] - \mathbb{E}[X_1^2]}
     {\mathbb{E}[X_1^2] - \mathbb{E}[X_1]^2} &=
\frac{
	\frac{\alpha_1}
		{\alpha_0}
	- \frac{\alpha_1(\alpha_1 + 1)}
		{\alpha_0(\alpha_0 + 1)}
	}
	{
	\frac{\alpha_1(\alpha_1 + 1)}
		{\alpha_0(\alpha_0 + 1)}
	- \frac{\alpha_1^2}
		{\alpha_0^2}
	}
\label{eq:5}
\\
&=
\frac{
    \frac{\alpha_1 (\alpha_0 + 1) - \alpha_1(\alpha_1 + 1)}
         {\alpha_0 (\alpha_0 + 1)}
}{
    \frac{\alpha_0 \alpha_1 (\alpha_1 + 1) - \alpha_1^2 (\alpha_0 + 1)}
         {\alpha_0^2 (\alpha_0 + 1)}
}
\\
&=
\frac{\alpha_0 \alpha_1 (\alpha_0 - \alpha_1)}
     {\alpha_1 (\alpha_0 \alpha_1 + \alpha_0 - \alpha_0 \alpha_1 - \alpha_1)}
\\
&=
\alpha_0
\\
\alpha_i &= \mathbb{E}[X_i] \alpha_0 \label{eq:6}
\end{align}

Z rovnice~(\ref{eq:5}) vypočítáme sumu všech parametrů $\alpha_0$. 
Protože střední hodnota proměnné z Dirichletovského rozdělení je právě
$\frac{\alpha_i}{\alpha_0}$, tak jednotlivé parametry získáme z
rovnice~(\ref{eq:6}).

\section{Algoritmus}

\begin{algorithm}
\caption{Expectation Propagation pro učení parametrů}
\label{alg:ep}
\begin{algorithmic}
\State Parametry zpráv z faktoru $\beta$ do množiny parametrů
    $\vec{\theta}_i$ označíme $\vec\alpha_{f_\beta \rightarrow \vec\theta_i}$.
\State Parametry zpráv z množiny parametrů $\vec{\theta}_i$ do faktoru $\beta$
    označíme $\vec\alpha_{ \vec\theta_i \rightarrow f_\beta}$.
\State Parametry marginální distribuce množiny parametrů $\vec{\theta}_i$
    označíme $\vec{\alpha}_i$.
\Init
    \State Nastav zprávy mezi faktory a proměnnými na $1$.
    \State Nastav parametry $\vec\alpha_{f_\beta \rightarrow \vec\theta_i}$ na $1$.
    \State Nastav parametry $\vec{\alpha}_i$ na apriorní hodnotu.
\EndInit
\Repeat
    \State Vyber faktor $f_{\tilde\beta}$, který se bude aktualizovat.

    \State Spočítej všechny zprávy z parametrů:
    \For{každý parametr $\vec{\theta}_i$ spojený s faktorem $f_{\tilde\beta}$}
        \State Parametry zprávy z $\vec{\theta}_i$ do $f_{\tilde\beta}$:
            $\vec\alpha_{ \vec\theta_i \rightarrow f_\beta} = \vec{\alpha}_i -
            \vec\alpha_{f_\beta \rightarrow \vec\theta_i } + 1$.
    \EndFor

    \State Aktualizuj zprávy z faktoru do proměnných:
    \For{každou proměnnou $X_i$, spojenou s faktorem $f_{\tilde\beta}$}
    \State Zpráva z $f_{\tilde\beta}$ do $X_i$ podle (\ref{eq:msgfromftox}):
    \State
        \hskip\algorithmicindent
        $\hat{f}(x_j) =
            \sum_{\vec{x}: x_j = \tilde{x}_j}
                \mathbb{E}_{q^\backslash}(\theta_{\rho(\vec{x^\prime}), x_0})
                \prod_{i \ne j}
                    m_{x_i \rightarrow f_{\tilde\beta}} (x_i)$
    \EndFor

    \State Aktualizuj marginální pravděpodobnost parametrů:
    \For{každý parametr $\vec{\theta}_i$ spojený s faktorem $f_{\tilde\beta}$}
        \State Spočítej parametry $\vec{\alpha}_i^*$ pro Dirichletovské
            rozdělení, které nejlépe
        \State aproximuje cílovou marginální distribuci
            (\ref{eq:marginaltheta}). Metoda popsána
        \State v předchozí sekci.
        \State Parametry zprávy z $f_{\tilde\beta}$ do $\vec{\theta}_i$:
        \State
            \hskip\algorithmicindent
            $\vec\alpha_{f_\beta \rightarrow \vec\theta_i} =
                \vec{\alpha}_i^*
                - \vec\alpha_{ \vec\theta_i \rightarrow f_\beta}
                + 1$
        \State Aktualizuj parametry marginální distribuce $q(\vec{\theta}_i)$
        \State
            \hskip\algorithmicindent
            $\vec{\alpha}_i = \vec{\alpha}_i^* =
                \vec\alpha_{f_\beta \rightarrow \vec\theta_i}
                + \vec\alpha_{ \vec\theta_i \rightarrow f_\beta}$
    \EndFor
    \For{každou proměnnou $X_i$, spojenou s faktorem $f_{\tilde\beta}$}
        \State Aktualizuj zprávy z proměnných do faktoru:
        \State
            \hskip\algorithmicindent
            $m_{x_i \rightarrow f_{\tilde\beta}}(x_i) =
                \prod_{\beta \ne \tilde\beta} m_{f_\beta \rightarrow x_i}(x_i)$
    \EndFor
\Until{konvergence}
\end{algorithmic}
\end{algorithm}
