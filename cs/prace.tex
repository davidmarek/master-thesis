%%% Hlavní soubor. Zde se definují základní parametry a odkazuje se na ostatní části. %%%

%% Verze pro jednostranný tisk:
% Okraje: levý 40mm, pravý 25mm, horní a dolní 25mm
% (ale pozor, LaTeX si sám přidává 1in)
\documentclass[12pt,a4paper]{report}
\setlength\textwidth{145mm}
\setlength\textheight{247mm}
\setlength\oddsidemargin{15mm}
\setlength\evensidemargin{15mm}
\setlength\topmargin{0mm}
\setlength\headsep{0mm}
\setlength\headheight{0mm}
% \openright zařídí, aby následující text začínal na pravé straně knihy
\let\openright=\clearpage

%% Pokud tiskneme oboustranně:
% \documentclass[12pt,a4paper,twoside,openright]{report}
% \setlength\textwidth{145mm}
% \setlength\textheight{247mm}
% \setlength\oddsidemargin{15mm}
% \setlength\evensidemargin{0mm}
% \setlength\topmargin{0mm}
% \setlength\headsep{0mm}
% \setlength\headheight{0mm}
% \let\openright=\cleardoublepage

%% Pokud používáte csLaTeX (doporučeno):
%\usepackage{czech}
% Pokud nikoliv:
\usepackage[czech]{babel}
\usepackage[T1]{fontenc}

%% Použité kódování znaků: obvykle latin2, cp1250 nebo utf8:
\usepackage[utf8]{inputenc}

%% Ostatní balíčky
\usepackage{tikz}
\usepackage{todonotes}
\usepackage{graphicx}
\usepackage{amsthm}
\usepackage{amsmath}
\usepackage{amsfonts}
\usepackage{epstopdf}
\usepackage{bm}
\usepackage{algpseudocode}
\usepackage{algorithm}
\usepackage{cite}
\usepackage{setspace}

%% Balíček hyperref, kterým jdou vyrábět klikací odkazy v PDF,
%% ale hlavně ho používáme k uložení metadat do PDF (včetně obsahu).
%% POZOR, nezapomeňte vyplnit jméno práce a autora.
\usepackage[unicode]{hyperref}   % Musí být za všemi ostatními balíčky
\hypersetup{pdftitle=Název práce}
\hypersetup{pdfauthor=David Marek}

%%% Drobné úpravy stylu

% Tato makra přesvědčují mírně ošklivým trikem LaTeX, aby hlavičky kapitol
% sázel příčetněji a nevynechával nad nimi spoustu místa. Směle ignorujte.
\makeatletter
\def\@makechapterhead#1{
  {\parindent \z@ \raggedright \normalfont
   \Huge\bfseries \thechapter. #1
   \par\nobreak
   \vskip 20\p@
}}
\def\@makeschapterhead#1{
  {\parindent \z@ \raggedright \normalfont
   \Huge\bfseries #1
   \par\nobreak
   \vskip 20\p@
}}
\makeatother

% Toto makro definuje kapitolu, která není očíslovaná, ale je uvedena v obsahu.
\def\chapwithtoc#1{
\chapter*{#1}
\addcontentsline{toc}{chapter}{#1}
}

\renewcommand{\vec}[1]{\bm{#1}}
\hyphenation{prav-dě-po-do-bno-sti}
\hyphenation{sdru-že-né}

\floatname{algorithm}{Algoritmus}
\algblockdefx[INIT]{Init}{EndInit}{\textbf{init}}{\textbf{end init}}

\usetikzlibrary{bayesnet}

\newtheorem{definice}{Definice}
\newtheorem{theorem}{Věta}


\begin{document}

% Trochu volnější nastavení dělení slov, než je default.
\lefthyphenmin=2
\righthyphenmin=2

%%% Titulní strana práce

\pagestyle{empty}
\begin{center}

\large

Univerzita Karlova v Praze

\medskip

Matematicko-fyzikální fakulta

\vfill

{\bf\Large DIPLOMOVÁ PRÁCE}

\vfill

\centerline{\mbox{\includegraphics[width=60mm]{../img/logo.eps}}}

\vfill
\vspace{5mm}

{\LARGE David Marek}

\vspace{15mm}

% Název práce přesně podle zadání
{\LARGE\bfseries Implementace aproximativních bayesovských metod pro odhad stavu v dialogových systémech}

\vfill

% Název katedry nebo ústavu, kde byla práce oficiálně zadána
% (dle Organizační struktury MFF UK)
Ústav formální a aplikované lingvistiky

\vfill

\begin{tabular}{rl}

Vedoucí diplomové práce: & Ing. Mgr. Filip Jurčíček, Ph.D. \\
\noalign{\vspace{2mm}}
Studijní program: & program \\
\noalign{\vspace{2mm}}
Studijní obor: & obor \\
\end{tabular}

\vfill

% Zde doplňte rok
Praha 2013

\end{center}

\newpage

%%% Následuje vevázaný list -- kopie podepsaného "Zadání diplomové práce".
%%% Toto zadání NENÍ součástí elektronické verze práce, nescanovat.

%%% Na tomto místě mohou být napsána případná poděkování (vedoucímu práce,
%%% konzultantovi, tomu, kdo zapůjčil software, literaturu apod.)

\openright

\noindent
Poděkování.

\newpage

%%% Strana s čestným prohlášením k diplomové práci

\vglue 0pt plus 1fill

\noindent
Prohlašuji, že jsem tuto diplomovou práci vypracoval(a) samostatně a výhradně
s~použitím citovaných pramenů, literatury a dalších odborných zdrojů.

\medskip\noindent
Beru na~vědomí, že se na moji práci vztahují práva a povinnosti vyplývající
ze zákona č. 121/2000 Sb., autorského zákona v~platném znění, zejména skutečnost,
že Univerzita Karlova v Praze má právo na~uzavření licenční smlouvy o~užití této
práce jako školního díla podle §60 odst. 1 autorského zákona.

\vspace{10mm}

\hbox{\hbox to 0.5\hsize{%
V ........ dne ............
\hss}\hbox to 0.5\hsize{%
Podpis autora
\hss}}

\vspace{20mm}
\newpage

%%% Povinná informační strana diplomové práce

\vbox to 0.5\vsize{
\setlength\parindent{0mm}
\setlength\parskip{5mm}

Název práce:
Implementace aproximativních bayesovských metod pro odhad stavu v dialogových systémech
% přesně dle zadání

Autor:
David Marek

Katedra:  % Případně Ústav:
Ústav formální a aplikované lingvistiky
% dle Organizační struktury MFF UK

Vedoucí diplomové práce:
Ing. Mgr. Filip Jurčíček, Ph.D., Ústav formální a aplikované lingvistiky
% dle Organizační struktury MFF UK, případně plný název pracoviště mimo MFF UK

Abstrakt:
% abstrakt v rozsahu 80-200 slov; nejedná se však o opis zadání diplomové práce

Klíčová slova:
% 3 až 5 klíčových slov

\vss}\nobreak\vbox to 0.49\vsize{
\setlength\parindent{0mm}
\setlength\parskip{5mm}

Title:
% přesný překlad názvu práce v angličtině

Author:
David Marek

Department:
Institute of Formal and Applied Linguistics
% dle Organizační struktury MFF UK v angličtině

Supervisor:
Ing. Mgr. Filip Jurčíček, Ph.D., Institute of Formal and Applied Linguistics
% dle Organizační struktury MFF UK, případně plný název pracoviště
% mimo MFF UK v angličtině

Abstract:
% abstrakt v rozsahu 80-200 slov v angličtině; nejedná se však o překlad
% zadání diplomové práce

Keywords:
% 3 až 5 klíčových slov v angličtině

\vss}

\newpage

%%% Strana s automaticky generovaným obsahem diplomové práce. U matematických
%%% prací je přípustné, aby seznam tabulek a zkratek, existují-li, byl umístěn
%%% na začátku práce, místo na jejím konci.

\openright
\pagestyle{plain}
\setcounter{page}{1}
\tableofcontents

%%% Jednotlivé kapitoly práce jsou pro přehlednost uloženy v samostatných souborech
\onehalfspacing
% !TEX root = prace.tex
\chapter*{Úvod}
\addcontentsline{toc}{chapter}{Úvod}

Dialog je přirozený způsob dorozumívání a sdělování informací mezi lidmi.
Počítač, který by dokázal vést dialog s uživatelem, byl vždy snem nejen
příznivců vědecko-fantastické literatury.
Už pro první počítače vnikaly programy, které se snažily využívat přirozenou
řeč pro interakci s uživatelem.
Jedním z takových programů byl například Eliza, program, který předstíral, že
jej zajímá, co mu uživatel říká.
Fungoval na principu rozpoznání textu pomocí gramatiky a následné transformace
textu do promluv dle pravidel.
Avšak gramatiky a pravidlové systémy se ukázaly nedostačné pro praktické
aplikace a tak se vývoj přesunul do statistických metod.
S využitím statistickým metod a metod strojového učení bylo možné začít s
porozumíváním mluveného slova.
Přijetí bylo zpočátku chladné a veřejnost byla 

\section{Cíle}

Dialogové systémy často využívají pouze nejlepší hypotézu ze systému porozumění přirozené řeči.
Většina ovšem umí vytvořit seznam $n$ nejlepších hypotéz.
Tato práce si klade za cíl představit metodu pro inferenci dialogového stavu v dialogovém systému s využitím více hypotéz.
Představená metoda bude založena na reprezentaci dialogového stavu pomocí dynamických bayesovských sítí.

Bude představen algoritmus Loopy Belief Propagation, který bude implementován pro využití v reálných systémech pro odhad dialogového stavu.
Algoritmus bude otestován na datech z reálného dialogového systému Let's Go a porovnán s dalšími systémy, které se účastnily soutěže Dialog State Tracking Challenge 2013.
Důležitou části algoritmů pro inferenci je určení pořadí, v jakém má inference probíhat.
Práce bude obsahovat implementaci několika strategií pro inferenci, které budou umožňovat efektivní inferenci pro různé druhy bayesovských sítí (stromy, dynamické sítě, obecné grafy).

Nakonec se práce bude zabývat učením parametrů sítě a představí algoritmus Expectation Propagation, který umožňuje inferenci v bayesovských sítích se spojitými náhodnými proměnnými.
Algoritmus Expectation Propagation je pouze konceptem, do kterého je třeba zasadit výpočet parametrů pravděpodobnostních rozložení z konkrétního grafického modelu.
Cílem práce není vytvořit obecný systém pro inferenci v libovolném grafickém modelu.
Takový cíl není ani reálný, protože při většině reálných použití dochází k aproximacím a úprávám v závislosti na ontologii problému, aby bylo možné inferenci provádět v reálném čase.
Cílem této práce tedy je vytvořit základní funkcionalitu potřebnou pro inferenci s EP, s kterou bude možné jednoduše vytvořit funkční systém, impolementovaný s doménově specifickými informacemi.
Jako příklad bude ukázán systém pro učení parametrů pravděpodobnostního rozložení pro pozorování v generickém modelu pro reprezentaci dialogového stavu.

% !TEX root = prace.tex
\chapter{Teorie dialogových systémů}
\label{ch:kap1}

\section{Dialogový systém}

Dialogový systém je počítačový systém, který umožňuje uživatelům komunikovat s počítačem ve formě, která je přirozená a efektivní pro použití.
Vývoj dialogových systémů má před sebou ještě spoustu problémů k překonání a pro praktické použití je třeba se uchýlit k několika předpokladům a zjednodušením.
Prvním zjednodušením je doménová specializace, v současné době není možné vytvořit dialogový systém, který by se dokázal s uživatelem bavit o libovolném tématu.
Z toho plyne i nutnost ontologie, systém potřebuje databázi popisující všechny informace, které může poskytnout a jejich strukturu.
Systém má definované fráze a věty, které může použít, na rozdíl od živého operátora, který dokáže improvizovat.

Další zjednodušení se týkají přímo dialogu.
Předpokládá se, že dialog probíhá vždy mezi systémem a jedním uživatelem.
Navíc se pravidelně střídají v obrátkách.
Jedna obrátka dialogu je složená z jedné promluvy systému a jedné promluvy uživatele.

Příkladem dialogového systému může být systém pro nalezení spojení pomocí městské dopravy.
Příkaz od uživatele může vypadat např. takto: \uv{Chci jet z Malostranského náměstí na Anděl}.
Dialogový systém z této věty odvodí, že uživatel hledá spojení, výchozí stanice je Malostranské náměstí a cílová stanice je Anděl.
Nyní záleží na chování dialogového systému, může například předpokládat, že uživatel chce vždy najít nejdřívější spojení, pokud neřekne jinak.
V takovém případě už systém může rovnou najít v databázi nejbližší spojení a uživateli jej sdělit.

Důležitou vlastností dialogového systému je robustnost.
Pokud budeme používat dialogový systém v přirozeném prostředí, musíme se vyrovnat s tím, že často nebude uživateli rozumět.
Systém může informaci přeslechnout, anebo si nemusí být jistý tím, co slyšel.
Dialogový systém se proto musí umět uživatele doptat na chybějící informace a musí umět pracovat s nejistotou.

\section{Součásti dialogového systému}

Dialogový systém se skládá z několika částí, které spolu komunikují.
Na vstupu je zvukový záznam uživatele, o jeho převedení do textu se stará systém rozpoznávání řeči (ASR).
Z textu je potřeba získat sémantické informace pomocí systému porozumění mluvené řeči (SLU).
Nad sémanticky anotovanými informacemi už může pracovat dialogový manager (DM), který zvolí patřičnou odpověď.
Výstupem dialogového manageru jsou informace, které se mají předat uživateli.
O jejich převedení do textu se stará systém generování přirozené řeči (NLG).
Do zvukového záznamu převede text syntetizér řeči (TTS).

\subsection{Systém rozpoznávání řeči (ASR)}

Systém rozpoznávání řeči slouží k převedení mluveného projevu do textové podoby.
Až po získání textové podoby je možné se zabývat významem textu.
Aktuálně nejlepší systémy jsou založené na pravděpodobnostním modelu a využívají Skryté Markovské modely (HMM) k určení nejpravděpodobnější sekvence slov pro daný zvukový záznam \cite{juang1991hidden}.
Pro tuto část dialogového systému existuje řada dostupných otevřených toolkitů, např. systém HTK \cite{young2002htk}, Kaldi \cite{Povey_ASRU2011} nebo SPHINX~\cite{walker2004sphinx}.
Existuje i celá řada komerčního software od firem jako IBM nebo Nuance.

Úspěšnost systému rozpoznávání řeči je závislá na obtížnosti úlohy a na počtu trénovacích dat, pocházejících ze stejné domény.
Pro obecnou doménu se problém stává mnohem těžší a je třeba velké množství dat.
Pro měření výkonu ASR systému se používá metrika Word Error Rate (WER).
Pro spočítání WER je třeba nejprve provést zarovnání rozpoznaného a originálního textu.
WER je pak počet slov, která jsou změněná, smazaná, anebo přidaná, vydělený počtem slov v originálním textu.Dialogový systém Let's Go!~\cite{raux2006doing} například dosahuje průměrné WER $64.3\%$.

Systém rozpoznávání řeči může produkovat více hypotéz pro jeden vstup.
Často existuje pro jeden zvukový záznam více možných slovních sekvencí, z kterých by mohl pocházet.
Reprezentace možných hypotéz může být seznam slovních sekvencí s jejich věrohodností.
Věrohodnosti jsou skóre přiřazené hypotézám, které určují jakou důvěru má systém rozpoznávání řeči ve správnost dané slovní sekvence.
Příklad možného seznamu hypotéz je v tabulce~\ref{tab:sezhyp}.

\begin{table}[h]
\begin{center}
\begin{tabular}{|c|l|}
\hline
Věrohodnost & Hypotéza \\
\hline
\hline
0.676 & poliklinika modřany\\
\hline
0.072 & poliklinika\\
\hline
0.063 & cheb poliklinika modřany\\
\hline
0.054 & poliklinika bory\\
\hline
0.045 & poliklinika modřany i\\
\hline
0.036 & polikliniky\\
\hline
0.027 & cheb poliklinika\\
\hline
0.018 & cheb poliklinika bory\\
\hline
0.009 & poliklinika i\\
\hline
\end{tabular}
\end{center}
\caption{Příklad seznamu hypotéz ze systému pro rozpoznávání přirozené řeči.
Uživatel by rád jel do zastávky Poliklinika Modřany.}
\label{tab:sezhyp}
\end{table}

Pro práci s hypotézami je vhodné, aby věrohodnost odpovídala aposteriorní pravděpodobnosti sekvence slov, dáno vstupní zvuk.

Další možností jak reprezentovat výstup je použití konfůzní sítě \cite{bertoldi2005new}.
Konfůzní síť je vážený orientovaný graf, obsahující startovní a konečný vrchol a hrany označené slovy.
Každá cesta ze startovního do konečného vrcholu vede přes všechny ostatní vrcholy.
Váhy hran jsou pravděpodobnosti slova přiřazeného dané hraně.
Hrany mohou obsahovat i prázdné slovo $\epsilon$.
Pravděpodobnost sekvence slov je součinem vah po cestě ze startovního do konečného uzlu.
Výhodou konfůzní sítě je, že umožňuje v komprimované podobě uložit mnohem více hypotéz.
Příklad konfůzní sítě je na obrázku~\ref{fig:konnet}.

\begin{figure}
\begin{center}
\begin{tikzpicture}[->, main/.style={circle, draw}]


\node[main] at (0,0) (n0) {};
\node[main] at (2.5,0) (n1) {};
\node[main] at (5.5,0) (n2) {};
\node[main] at (9.5,0) (n3) {};
\node[main] at (12.5,0) (n4) {};


\node at (1.25, 1) (a0) {0.90, jet};
\node at (1.25, 0.2) (a1) {0.065, $\epsilon$};
\node at (1.25, -1) (a2) {0.035, jak};

\path[looseness=0.7, out=75, in=105] (n0) edge (n1);
\path (n0) edge (n1);
\path[looseness=0.6, bend right, out=285, in=255] (n0) edge (n1);

\node at (4, 2) (b0) {0.65, s};
\node at (4, 1) (b1) {0.25, se};
\node at (4, 0.2) (b2) {0.015, ses};
\node at (4, -1) (b3) {0.025, dnes};
\node at (4, -2) (b4) {0.06, $\epsilon$};

\path[looseness=1.8, out=90, in=90] (n1) edge (n2);
\path[looseness=0.7, out=75, in=105] (n1) edge (n2);
\path (n1) edge (n2);
\path[looseness=0.6, bend right, out=285, in=255] (n1) edge (n2);
\path[looseness=1.7, bend right, out=270, in=270] (n1) edge (n2);

\node at (7.5, 2) (c0) {0.61, mnichovské};
\node at (7.5, 1) (c1) {0.25, smíchovskou};
\node at (7.5, 0.2) (c2) {0.04, mnichovského};
\node at (7.5, -1) (c3) {0.03, mnichovská};
\node at (7.5, -2) (c4) {0.07, $\epsilon$};

\path[looseness=1.3, out=90, in=90] (n2) edge (n3);
\path[looseness=0.5, out=75, in=105] (n2) edge (n3);
\path (n2) edge (n3);
\path[looseness=0.5, bend right, out=285, in=255] (n2) edge (n3);
\path[looseness=1.3, bend right, out=270, in=270] (n2) edge (n3);

\node at (11, 0.2) (d0) {1.0, nádraží};
\path (n3) edge (n4);

\end{tikzpicture}
\caption{Příklad konfůzní sítě.}
\label{fig:konnet}
\end{center}
\end{figure}

\subsection{Porozumění mluvené řeči (SLU)}

Po získání hypotéz o promluvě uživatele se musi dialogový systém pokusit porozumět, co se uživatel snažil sdělit.
Dialogový systém nepotřebuje vědět, co přesně uživatel řekl, důležité je pouze zjistit význam sdělení.
Pokud například uživatel řekne "Chtěl bych nalézt spojení z Malostranského náměstí na Anděl", anebo "Jak se dostanu na Anděl ze zastávky Malostranské náměstí?", tak pro dialogový systém jsou obě tvrzení ekvivalentní, uživatel požaduje informace o spojení mezi dvěma zastávkami, i když v jednom případě jde o větu oznamovací a v druhém případě o otázku.

První rozdělení mezi přesným významem sdělení a jeho účelem se objevilo ve formě tzv. speech aktů~\cite{austin1975things}.
Následně byla idea speech aktů rozšířena pro potřeby dialogových systémů~\cite{traum1999speech} a výsledek byl nazván dialogovým aktem.

V této práci používáme definici dialogového aktu, která byla použita v soutěži Dialog State Tracking Challenge~\cite{williamsdialog}.
Semántická reprezentace sdělení uživatele se tedy nazývá dialogový akt (DA), skládá se z jedné nebo více položek dialogového aktu (DAI), které jsou spojené v konjunkci.
Každá DAI se skládá z typu, názvu slotu a jeho hodnoty. Typy jsou doménově nezávislé, sloty a jejich hodnoty reprezentují koncepty ontologie.
Příklad dialogového aktu ze systému pro hledání spojení v městské dopravě:

\begin{center}
{\tt hello(), inform(route="61a")}.
\end{center}

Zde se dialogový akt skládá ze dvou položek, první položka má pouze typ {\em hello}, značící pozdrav.
Druhá položka má typ {\em inform}, tzn. uživatel nás informuje o svém požadavku.
Název slotu je {\em route} a hodnota je \uv{61a}, tedy uživatel nám říká, že hledá spojení linkou 61a.

Typů může být libovolné množství, ale existuje několik základních, jejichž použití je ustálené.
\begin{itemize}
\item {\em inform} --- sdělujeme informaci, doplňujeme hodnotu do slotu,
\item {\em request} --- požadujeme od protějšku doplnění hodnoty pro dotazovaný slot,
\item {\em confirm} --- chceme potvrdit hodnotu slotu, potvrzení může být implicitní, anebo explicitní.
	Při explicitním potvrzení očekáváme odpověď \uv{Ano} nebo \uv{Ne},
	U implicitního, pokud se nám nedostane odpovědi předpokládáme, že protějšek souhlasí.
\end{itemize}

V tabulce~\ref{tab:dstcdat} jsou ukázány všechny typy dialogových aktů, které může uživatel říct v dialogovém systému Let's Go!~\cite{williamsdialog}.

\begin{table}
\begin{center}
\begin{tabular}{|l|}
\hline
Typ dialogového aktu \\
\hline
\hline
hello \\
\hline
bye \\
\hline
goback \\
\hline
restart \\
\hline
null \\
\hline
repeat \\
\hline
nextbus \\
\hline
prevbus \\
\hline
tellchoices \\
\hline
affirm \\
\hline
negate \\
\hline
deny \\
\hline
inform \\
\hline
\end{tabular}
\caption{Typy dialogových aktů, které může říct použít uživatel v systému Let's Go!}
\label{tab:dstcdat}
\end{center}
\end{table}

Existuje široké množství technik, které lze použít pro porozumění mluvené řeči.
Unifikace pomocí šablon anebo gramatiky jsou příklady ručně psaných metod.
Metody založené na datech jsou například Hidden Vector State model~\cite{he2005semantic}, techniky strojového překladu~\cite{wong2007learning}, Combinatory Categorical Grammars~\cite{zettlemoyer2007online} nebo Support Vector Machines~\cite{mairesse2009spoken}.

\subsection{Dialogový manager (DM)}

Pokud už jsou pravděpodobné dialogové akty dekódovány, je třeba rozhodnout, jak bude systém reagovat.
Komponenta tvořící rozhodnutí o dalším kroku systému se nazývá dialogový manager.
Odpověď systému je zakódována do formy dialogových aktů a nazývá se systémová akce.

Zvolená systémová akce je vybrána z množiny možných akci $a \in \mathcal{A}$ a závisí na vstupu, který systém obdržel z SLU.
Tento vstup se nazývá pozorování $o \in \mathcal{O}$, protože obsahuje vše, co systém pozoroval o uživateli.

Zvolení správné akce potřebuje více znalostí než jen poslední pozorování.
Celá historie dialogu a také kontext hrají důležitou roli.
Dialogový manager bere na vše ohled pomocí udržovaní interní reprezentace celého pozorovaného dialogu.
Tato reprezentace se nazývá dialogový stav, nebo také stav důvěry, značí se $b \in \mathcal{B}$.
Aktuální dialogový stav závisí na přechodové funkci, která dialogový stav aktualizuje pro každé nové pozorování a systémovou akci.
Přechodová funkce je tedy mapování $\mathcal{T} : \mathcal{B} \times \mathcal{A} \times \mathcal{O} \longrightarrow \mathcal{B}$.
V této práci se budeme věnovat právě metodám aktualizace dialogového stavu.

Chování dialogového stavu definuje dialogová strategie $\pi$.
Strategie určuje co má systém provést v závislosti na aktuálním dialogovém stavu.
Obecně strategie vytvoří pravděpodobnostní rozložení přes možné akce.
Pokud $\prod(\mathcal{A})$ značí množinu těchto distribucí, pak dialogová strategie bude zobrazení z dialogového stavu do této množiny, $\pi: \mathcal{B} \longrightarrow \prod(\mathcal{A})$.

Pozorování, dialogový stav a akce jsou číslovány podle obrátky.
Pokud je časový okamžik důležitý, jsou pozorování, dialogový stav a akce z obrátky číslo $t$ označeny $o_t$, $b_t$ a $a_t$.

\subsection{Generování přirozené řeči (NLG a TTS)}

Posledním krokem dialogového systému je vytvoření odpovědi pro uživatele.
Nejprve systém generování přirozené řeči (NLG) převede dialogové akty na text.
Následně je text převeden na zvuk pomocí textového syntetizéru řeči (TTS).

Nejjednodušším přístup ke generování přirozeného jazyka z dialogových aktů je použití šablon.
Například pro dialogový akt {\tt inform(from.stop="x")} bude vytvořena šablona \uv{Pojedete ze zastávky x}, kde \uv{x} bude nahrazeno například za \uv{Malostranská}, \uv{Letňany}, atd.
Šablony jsou jednoduché a často efektivní řešení, protože počet možných frází je většinou dostatešně malý.

Při syntéze řeči existuje mnoho alternativ.
Je možné použít segmenty řeči z databáze pro vygenerování zvuků tvořících dohromady celou sekvenci slov.
Příkladem těchto systémů je Festival~\cite{black2001festival} nebo jeho odnož  FLite~\cite{black2001flite}.

Alternativní metodou syntézy je použití Skrytých Markovských modelů pro generování zvuku, příkladem je HTS systém~\cite{zen2007hmm}.

Pro syntézu řeči existují také komerční systémy, například systém SpeechTech TTS~\cite{speechtech}.

\section{Dialogový stav}

Dialogový stav je reprezentace všech informací, které lze o aktuálním dialogu získat.
To znamená všechno co uživatel a systém řekl.
Ovšem nejistota v tom, co vlastně uživatel řekl a jaké jsou jeho cíle, je základním problémem, s kterým se dialogový systém musí vypořádat.
Systémy pro rozpoznávání i porozumění řeči často chybují a tuto možnost musí brát dialogový manager také v potaz.
Vypořádat se s nejistotou lze pomocí jejího zakomponování do modelu pro odhad dialogového stavu.

Cíle uživatele a další vlastnosti prostředí lze považovat za náhodné částečně pozorovatelné proměnné a je možné je odvodit z pozorování.
Pravděpodobnostní rozložení těchto náhodných proměnných dává dobře definovanou reprezentaci nejistoty, navíc je možné je reprezentovat pomocí Bayesovské sítě.

Jedním z možných modelů dialogového stavu je generativní model.
Definujeme množinu dialogových stavů, $s \in \mathcal{S}$.
Předpokládáme, že pozorování závisí podmíněně pouze na stavu prostředí a definujeme pravděpodobnostní rozdělení pro pozorování, $p(o \mid s)$.
Dále předpokládáme, že cíle uživatele se nemění v čase a stav prostředí je tedy závislý pouze na stavu v předchozí obrátce a na poslední akci systému.
Tato závislost je zachycena v přechodové pravděpodobnosti $p(s_{t+1} \mid s_t, a_t)$.
Předpoklad, že dialogový stav závisí pouze na minulé hodnotě se nazývá Markovská vlastnost.

Pokud vezmeme předchozí předpoklady, tak lze využít bayesovský přístup pro počítání s nejistotou.
Stav v čase $t$ označíme $s_t$.
Podle Bayesova vzorce můžeme spočítat pravděpodobnost stavu v čase $t+1$ po přijetí nového pozorování $o_{t+1} = o^\prime$.

\begin{equation}
p(s_{t+1} = s^\prime) \propto
    \sum_{s \in \mathcal{S}}
        p(s_t = s)
        p(s_{t+1} = s^\prime \mid s_t = s, a_t = a) p(o_{t+1} = o^\prime \mid s_{t+1} = s^\prime)
\label{eq:beliefupdate}
\end{equation}

Nyní můžeme definovat stav důvěry v čase $t$, $b_t$, jako pravděpodobnost přes stavy dáno všechna pozorování až do času $t$.
Množina všech možných stavů důvěry je pravděpodobnost přes všechny možné dialogové stavy $\mathcal{B} = \prod(\mathcal{S})$.

Můžeme přepsat rovnici~\eqref{eq:beliefupdate} s pomocí stavů důvěry.

\begin{equation}
b(s_{t+1}) \propto
    \sum_{s \in \mathcal{S}}
        b(s_t)
        p(s_{t+1} \mid s_t, a_t)
        p(o_{t+1} \mid s_{t+1})
\label{eq:beliefupdate2}
\end{equation}

Rovnice~\eqref{eq:beliefupdate2} nám dává předpis pro přechodovou funkci $\mathcal{T}$.
V praxi ovšem bude množina možných hodnot pro stav $s_t$ příliš velká, protože stav musí obsahovat všechny informace potřebné pro rozhodování, to znamená celou historii dialogu a cíle uživatele. 
Pokud systém obsahuje sloty, tak každá kombinace hodnot slotu je jedním možným cílem uživatele. 
Tedy velikost stavového prostoru roste exponenciálně.

\subsection{Aktualizace stavu}

Efektivní metodou pro aktualizaci dialogového stavu je použití dynamických bayesovských sítí~\cite{thomson2008bayesian}.
Bayesovské sítě umoňují efektivní výpočet využitím podmíněných nezávislostí mezi sloty.
Stále ovšem zůstává problém s výpočtem, pokud i jednotlivé sloty obsahují příliš mnoho hodnot.
Lze použít aproximace a počítat jen s $k$ nejpravděpodobnějšími hodnotami~\cite{thomson2010bayesian}.

Alternativním zjednodušením je rozdělit stav prostředí do skupin. 
Tento přístup se nazývá Hidden Information State (HIS)~\cite{young2010hidden}.
Základním předpokladem zde musí být, že uživatel nezmění svůj cíl v průběhu dialogu.
Pak lze efektivně provádět aktualizaci, protože rovnice pro aktualizace pravděpodobnosti se nemění mezi jednotlivými skupinami.

V této práci se budeme zabývat prvním přístupem, tedy použitím Bayesovských sítích.
Pro inferenci použijeme Loopy Belief Propagation (LBP) algoritmus, který je aproximativní metodou pro sítě s diskrétními náhodnými proměnnými.
Pro učení parametrů představíme Expectation Propagation (EP) algoritmus.
EP je zobecněním LBP na libovolné pravděpodobnostní rozložení.

% !TEX root = prace.tex
\chapter{Inference v grafických modelech}

% !TEX root = thesis.tex
\chapter{Učení parametrů}
\label{ch:ep}

Představená metoda LBP v~minulé kapitole funguje pro modely, kde máme nastavené parametry faktorů.
V~této kapitole si představíme grafický model pro diskrétní proměnné, u~kterého jsou i parametry faktorů proměnné a je možné pro ně použít Bayesovský přístup.
Díky tomu bude možné nalézt aposteriorní distribuci pro tyto parametry a naučit je podle dat.

Inferenci v~tomto upraveném modelu už není možné dělat pomocí LBP a~budeme muset použít algoritmus Expectation Propagation.

\section{Grafický model}

Model se skládá z~faktorů a proměnných.
Nechť máme vybraný faktor $f_\beta$, spojený s~několika proměnnými
$\vec{x} = (x_0, x_1, \dots, x_{N_x})$
a množinami parametrů
$\vec{\Theta} = (\vec{\theta}_1, \dots, \vec{\theta}_{N_\theta})$.
Tento faktor reprezentuje podmíněnou pravděpodobnost:
$$f_\beta(\vec{x}, \Theta) = p(x_0 | x_1, \dots, x_{N_x}; \Theta).$$

Rodičovské proměnné $x_1, \dots, x_{N_x}$ označujeme jako $\vec{x^\prime}$.
Vektor $\vec{x^\prime}$ určuje, jaká množina parametrů bude použita.
Protože množiny parametrů jsou číslovány $1, \dots, N_\theta$ a rodičovské
proměnné $1, \dots, N_x$, musí být pro vybrání správné množiny parametrů
použito mapování $\rho(\vec{x^\prime})$.
Faktor pak může být zapsán zkráceně:
$$f_\beta(\vec{x}, \Theta) = p(x_0 | x_1, \dots, x_{N_x}; \Theta) =
\theta_{\rho(x^\prime), x_0}$$

Příklad faktoru $f_\beta$ je na obrázku~\ref{ex:factor}, pro ilustraci pouze s~třemi parametry a~čtyřmi proměnnými.
Správně by měl faktor mít tolik parametrů, kolik je možných přiřazení pro rodičovské proměnné.

\begin{figure}[h]
\begin{center}
\begin{tikzpicture}[->, main/.style={circle, draw}, factor/.style={draw}]
    
\node[main] at (-3.7, 2)    (t1) {$\vec\theta_1$};
\node[main] at (-4, 0)      (t2) {$\vec\theta_2$};
\node[main] at (-3.7, -2)   (t3) {$\vec\theta_3$};
\node[factor] at (0, 0)     (f)  {$f_\beta$};
\node[main] at (3.7, 2)     (x1) {$X_1$};
\node[main] at (4, 0)       (x2) {$X_2$};
\node[main] at (3.7, -2)    (x3) {$X_3$};
\node[main] at (0, -4)      (x0) {$X_0$};

\path
	(t1) edge [above right, near start] node {$q^{\backslash \beta}(\vec\theta_1)$} (f)
	(t2) edge [above, near start] node {$q^{\backslash \beta}(\vec\theta_2)$} (f)
	(t3) edge [below right, near start] node {$q^{\backslash \beta}(\vec\theta_3)$} (f)
	(x1) edge [above left, near start] node {$q^{\backslash \beta}(X_1)$} (f)
	(x2) edge [above, near start] node {$q^{\backslash \beta}(X_2)$} (f)
	(x3) edge [below left, near start] node {$q^{\backslash \beta}(X_3)$} (f)
	(f) edge [right, near end] node {$m_{f_\beta \rightarrow X_0}(X_0)$} (x0);


\end{tikzpicture}
\end{center}
\caption{Vybraný faktor $f_\beta$ pro aktualizaci při učení parametrů.}
\label{ex:factor}
\end{figure}

\section{Výpočet marginálních pravděpodobností}

Pro výpočet sdružené pravděpodobnosti používáme plně faktorizovanou distribuci.
Pro každou proměnnou (množinu parametrů) je její marginální pravděpodobnost
rovna součinu zpráv ze sousedních faktorů.
Pro daný faktor $f_\beta$ je neúplná distribuce $q^{\backslash \beta}(x_i)$, popř.
$q^{\backslash \beta}(\vec{\theta}_i)$, rovna součinu zpráv ze všech ostatních faktorů do $x_i$, popř $\vec\theta_i$.
Aproximovaná marginální pravděpodobnost proměnné (množiny parametrů) je pak součinem neúplné
distribuce a zprávy z~faktoru:
$$q(x_i) = q^{\backslash \beta}(x_i) m_{f_\beta \rightarrow x_i}(x_i), 
\quad \quad
q(\vec{\theta}_i) = q^{\backslash \beta}(\vec{\theta}_i) m_{f_\beta \rightarrow \theta}(\vec{\theta}_i)$$

Tyto neúplné distribuce jsou právě zprávy z~proměnné, popř. množiny parametrů, do faktoru:

$$m_{x_i \rightarrow f_\beta} = q^{\backslash \beta}(x_i), 
\quad \quad
m_{\vec\theta_i \rightarrow f_\beta} = q^{\backslash \beta}(\vec\theta_i).$$

\subsection{Marginální pravděpodobnost proměnných}

Pokud chceme aktualizovat hodnotu naší aproximace marginální pravděpodobnosti,
je třeba minimalizovat její vzdálenost od skutečné marginální
pravděpodobnosti, ve tvaru
\begin{align}
p^*(\tilde{x}_j) &=
\sum_{\vec{x} \backslash x_j} \,
	\int_{\vec{\Theta}} \,
    		\prod_i \,
			q^{\backslash \beta}(x_i) \,
		\prod_l \,
			q^{\backslash \beta}(\vec{\theta}_l) \;
		f_\beta(\vec{x};\,
    		  \vec{\Theta})
\label{eq:1} 
\\
&=
\sum_{\vec{x} \backslash x_j} \,
	\prod_i \,
		q^{\backslash \beta}(x_i) \,
    \int_{\vec{\theta_{\rho(\vec{x^\prime})}}} \,
	    q^{\backslash \beta}(\vec{\theta}_{\rho(\vec{x^\prime})})\;
    \theta_{\rho(\vec{x^\prime}), x_0} \label{eq:2} 
\\
&= 
\sum_{\vec{x} \backslash x_j} \,
	\prod_i \,
		q^{\backslash \beta}(x_i)\;
    		\mathbb{E}_{q^{\backslash \beta}} 
			(\theta_{\rho(\vec{x^\prime}), x_0}).
\label{eq:3}
\end{align}

Rovnost~(\ref{eq:1}) vychází z~definice výpočtu marginální pravděpodobnosti ze
sdružené pravděpodobnosti.
V~(\ref{eq:2}) byla použita definice faktoru, z~integrálu byly vytknuty členy neobsahující $\Theta$ a nakonec bylo využito toho, že pro
množiny parametrů, které nejsou spojeny s~faktorem $f_\beta$, je jejich neúplná
distribuce rovna marginální distribuci, a tedy $\int_{\theta_i} q(\theta_i) =
1$. V~(\ref{eq:3}) byla použita definice očekávané hodnoty.

Marginální pravděpodobnost proměnné $x_j$ tedy je
\begin{equation}
p^*(x_j) =
\sum_{\vec{x} \backslash x_j} \;
	\prod_i \;
		q^{\backslash \beta}(x_i)\;
    		\mathbb{E}_{q^{\backslash \beta}} 
			(\theta_{\rho(\vec{x^\prime}), x_0}).
\end{equation}

Tady docházíme k~výsledku, který je velmi podobný výpočtu marginální
pravděpodobnosti v~Loopy Belief Propagation algoritmu.

Zprávu z~faktoru $f_\beta$ do vrcholu $x_j$ pak získáme vydělením zprávy z~$x_j$ do $f_\beta$ z~marginální pravděpodobnosti,

\begin{equation}
m_{f_\beta \rightarrow x_j}(x_j) =
    \sum_{\vec{x} \backslash x_j} \;
        \prod_{i \ne j}\;
            q^{\backslash \beta}(x_i)\;
            \mathbb{E}_{q^{\backslash \beta}}
                (\theta_{\rho(\vec{x^\prime}), x_0}).
\label{eq:msgfromftox}
\end{equation}

\subsection{Marginální pravděpodobnost parametrů}

Pro množiny parametrů se jejich marginální pravděpodobnost spočítá podobně jako
pro proměnné:
\begin{align}
p^*(\tilde{\vec{\theta}}_j) & = 
    \sum_{\vec{x}} \; 
    \int_{\vec{\Theta} \backslash \vec{\theta}_j} \;
         \prod_i \;
             q^{\backslash \beta}(x_i) \;
         \prod_l \;
    q^{\backslash \beta}(\vec{\theta}_l) \;
    f_\beta(\vec{x}; \vec{\Theta}) \label{eq:ep:theta_1}
\\
& = 
    \sum_{l \ne j} \;
        \sum_{\vec{x}: \rho(\vec{x^\prime}) = l} \;
            \prod_i \;
                q^{\backslash \beta}(x_i) \,
                \int_{\vec{\Theta} \backslash \vec{\theta}_j} \;
                    \prod_k \;
                        q^{\backslash \beta}(\vec{\theta}_k) \;
                        \theta_{l, x_0}\; + \label{eq:ep:theta_2}
\\
& + 
    \sum_{\vec{x}: \rho(\vec{x^\prime}) = j} \;
        \prod_i \;
            q^{\backslash \beta}(x_i) \;
            \int_{\vec{\Theta} \backslash \vec{\theta}_j} \;
                \prod_k \;
                    q^{\backslash \beta}(\vec{\theta}_k) \; 
                    \tilde{\theta}_{j, x_0}
\nonumber
\\
& = 
    \left[ \;
        \sum_{l \ne j} \;
            \sum_{\vec{x}: \rho(\vec{x^\prime}) = l} \;
                \prod_i \;
                    q^{\backslash \beta}(x_i) \;
                    \mathbb{E}_{q^{\backslash \beta}(\vec{\theta}_l)} (\theta_{l, x_0}) \;
    \right] \;
    q^{\backslash \beta}(\tilde{\vec{\theta}}_j) \; + \label{eq:ep:theta_3}
\\
& + 
    \sum_{\vec{x}: \rho(\vec{x^\prime}) = j} \;
        \prod_i \;
            q^{\backslash \beta}(x_i) \;
            \tilde{\theta}_{j,x_0} \;
            q^{\backslash \beta}(\tilde{\vec{\theta}}_j) \;
\label{eq:ep:theta_4}
\nonumber
\\
& = w_0 \; q^{\backslash \beta}(\tilde{\vec{\theta}}_j) \, + \, \sum_k \; w_k \;
    \tilde{\theta}_{j,k} \; q^{\backslash \beta}(\tilde{\vec{\theta}}_j),
\end{align}

kde

\begin{align}
w_0 &=
	\sum_{l \ne j} \;
		\sum_{\vec{x}: \rho(\vec{x^\prime}) = l} \;
			\prod_i \;
 				q^{\backslash \beta}(x_i) \;
				\mathbb{E}_{q^{\backslash \beta}(\vec{\theta}_l)} (
					\theta_{l,
					x_0}) 
\\
w_k &=
	\sum_{\vec{x}: \rho(\vec{x^\prime}) = j, x_0 = k} \; 
		\prod_i \;
			q^{\backslash \beta}(x_i)		
\end{align}

Opět vycházíme z~výpočtu marginální pravděpodobnosti podle definice.
V~rovnici \eqref{eq:ep:theta_2} jsme rozdělili sumu přes
$\vec{x}$ na ty, pro které se ve faktoru použije množina parametrů
$\tilde{\vec{\theta}}_j$, a na ty ostatní. Také jsme z~integrálu vytknuli součin
neúplných distribucí pro proměnné. V~dalším kroku \eqref{eq:ep:theta_3} jsme opět
použili toho, že integrál přes $\vec{\Theta}$ je ve skutečnosti několik
integrálů přes jednotlivé množiny parametrů. Díky tomu je můžeme vložit mezi členy produktu neúplných distribucí pro množiny parametrů. Ve výsledku
získáme $q^{\backslash \beta}(\tilde{\vec{\theta}}_j) \, \int_{\vec{\theta}_l} \,
q^{\backslash \beta}(\vec{\theta}_l) \; \theta_{l, x_0}$ a pak zbylé členy, které zmizí.

Docházíme k~vyjádření skutečné marginální pravděpodobnosti, v~níž není třeba
integrovat přes všechny množiny parametrů, ale stačí jen očekávaná hodnota
těchto parametrů.

\section{Aproximace marginálních pravděpodobností}

Při výpočtu aproximující distribuce $q(\vec{\theta}_j)$ se může stát, že skutečná marginální distribuce je v~komplikovaném tvaru a pak je aproximace složitá na výpočet a může se odchylovat od skutečné distribuce.
Například se jedná o~směs distribucí.
Ideální případ je takový, kdy aproximace je stejného typu jako skutečná distribuce.

Pro zjednodušení výpočtů, jak se dále ukáže, zvolíme zprávy z~faktoru do množiny parametrů, $m_{f \rightarrow \vec{\theta}_i}(\vec{\theta}_i)$, ve tvaru Dirichletovské distribuce s~parametry $\vec\alpha_{f \rightarrow \vec\theta_i}$:
\begin{equation}
m_{f \rightarrow \vec{\theta}_i}(\vec{\theta}_i) =
	Dir(\vec{\theta}_i; \vec\alpha_{f \rightarrow \vec\theta_i}) =
        \frac{\Gamma (\sum_j \vec\alpha_{f \rightarrow \vec\theta_i, j})}
             {\prod_j \Gamma(\vec\alpha_{f \rightarrow \vec\theta_i, j})}
        \prod_j \theta_{i,j}^{\vec\alpha_{f \rightarrow \vec\theta_i, j} - 1},
\end{equation}
kde $\Gamma$ je Gamma funkce (zobecnění faktoriálu):
\begin{equation}
    \Gamma(z) = \int_0^\infty \! t^{z-1} \exp(-t) \;\mathrm{d}t.
\end{equation}

Dirichletovská distribuce byla zvolena, protože má důležité vlastnosti pro
součin, které budou využity dále pro výpočet neúplné distribuce a celkové
aproximace. Pokud označíme aproximované faktory indexem $\beta$ a každý bude
mít vlastní parametry $\vec\alpha_{f_\beta \rightarrow \vec\theta_i}$, tak výsledná aproximace bude
tvaru:
\begin{align}
q(\vec{\theta}_i) \,& \, \propto
    \prod_\beta \;
    m_{f_\beta \rightarrow \vec{\theta}_i}(\vec{\theta}_i) \,
 \propto \,
    \prod_\beta \;
        \prod_j \;
            \theta_{i,j}^{\vec\alpha_{f_\beta \rightarrow \vec\theta_i, j} \,-\, 1}
\\
&\propto \,
    Dir(\vec{\theta}_i ;\, \sum_\beta \vec\alpha_{f_\beta \rightarrow \vec\theta_i} - (\,|\beta| - 1\,) \, \vec{1})
\\
&=
    Dir(\vec{\theta}_i;\,
        \vec{\alpha}_i),
\label{eq:ep:aprx_1}
\end{align}
kde $\vec{\alpha}_i = \sum_\beta \vec\alpha_{f_\beta \rightarrow \vec\theta_i} - (\,|\beta| - 1\,)
\vec{1})$.

Při aktualizaci faktoru $\tilde\beta$ tedy neúplná distribuce bude
\begin{align}
q^{\backslash \tilde\beta} (\vec{\theta_i})
&\propto
    \prod_{\beta \ne \tilde\beta}
        m_{f_\beta \rightarrow \vec{\theta}_i}(\vec{\theta}_i)
\\
& \propto
Dir(\vec{\theta}_i;
    \vec{\alpha}_i - \vec\alpha_{f_\beta \rightarrow \vec\theta_i} + \vec{1}).
\end{align}

Naším cílem je nalézt parametry $\vec{\alpha}^*$ aproximované marginální pravděpodobnosti \eqref{eq:ep:aprx_1}, které minimalizují vzdálenost od skutečné marginální pravděpodobnosti \eqref{eq:ep:theta_4}. 

Jednotlivé faktory musíme vždy aproximovat s~ohledem na všechny ostatní.
U~našeho grafického modelu využijeme nezávislostí, a tak nám stačí pouze pracovat se zprávami ze sousedních vrcholů.
Všechny informace se v~modelu šíří přes zprávy mezi vrcholy, a proto je třeba několika iterací pro přesun informace z~jedné části modelu do druhé.

Pro měření vzdálenosti mezi dvěma pravděpodobnostními rozloženími se používá Kullback-Leiblerova divergence:
\begin{equation}
KL(p \| q) =
\int_{-\infty}^{\infty}
    p(x) \log\left(\frac{p(x)}{q(x)}\right) \mathrm{d}x
\end{equation}
Pro nalezení minima použijeme algoritmus srovnání momentů a budeme
minimalizovat $KL(p^*\| q)$.

Pokud se podíváme na skutečnou marginální pravděpodobnost
$p^*(\vec{\theta}_i)$, zjistíme, že můžeme některé její členy upravit.
Využijeme také vlastnosti Gamma funkce $\Gamma(x) = (x-1) \Gamma(x-1)$.

\begin{align}
w_j \, \theta_j \, Dir(\vec{\theta}; \vec{\alpha}) &\propto \,
    w_j \, \theta_j \, \frac{\Gamma (\, \sum_i \, \alpha_{i} \,)}{\prod_i \, \Gamma(\,\alpha_i\,)}\;
    \prod_i \, \theta_{i}^{\alpha_{i} - 1}
\label{eq:4}
\\
&\propto w_j \, \frac{\Gamma (\,\sum_i \, \alpha_{i}\,)} {\prod_i \,
    \Gamma(\,\alpha_{i}\,)} \; \theta_j^{\alpha_j} \prod_{i \ne j} \,
    \theta_{i}^{\alpha_{i} - 1} \\
&\propto w_j \,
    \frac{\Gamma (\,\sum_i \alpha_{i}\,)}
         {\prod_i \, \Gamma(\, \alpha_{i} \,)} \;
    \frac{\Gamma(\, \alpha_{j} + 1 \,) \; \prod_{i \ne j} \, \Gamma(\,\alpha_i\,)}
         {\Gamma (\,1 + \sum_i \alpha_i\,)} \;
    Dir(\vec\theta; \, \vec{\alpha} + \vec{\delta}_j) \\
&\propto w_j \,
    \frac{\Gamma (\,\sum_i \alpha_{i}\,)}
         {\prod_i \, \Gamma(\, \alpha_{i}\,)} \;
    \frac{\alpha_j \, \Gamma(\,\alpha_j\,) \, \prod_{i \ne j} \, \Gamma(\,\alpha_i\,)}
         {(\,\sum_i \, \alpha_i\,) \, \Gamma (\,\sum_i \,\alpha_i\,)}\;
    Dir(\vec\theta; \,\vec{\alpha} + \vec{\delta}_j) \\
&\propto w_j \,
    \frac{\alpha_j}
         {\sum_i \, \alpha_i} \;
    Dir(\vec\theta; \,\vec{\alpha} + \vec{\delta}_j) \\
\end{align}

Zde jsme použili $\vec{\delta}_i$, což je vektor nul, který má pouze na pozici $i$ jedničku.

Díky této úpravě lze $p^*$ vyjádřit jako směs Dirichletovských rozdělení.
\begin{equation}
p^*(\vec{\theta}) =
    w_0^* \; Dir(\vec{\theta};\, \vec{\alpha}) +
    \sum_j \, w^*_j \;
        Dir(\vec{\theta};\, \vec{\alpha} + \vec{\delta}_j),
\label{eq:marginaltheta}
\end{equation}
kde
\begin{align}
    w^*_0 \propto w_0, \quad \quad
    w^*_j \propto w_j \, \frac{\alpha_j}{\sum_i \alpha_i}, \quad \quad
    \sum_{i=0}^k w_i^* = 1.
\end{align}

Jak bylo ukázáno v~sekci~\ref{sec:expfam}, pro minimalizaci KL divergence mezi dvěma rozděleními z~exponenciální rodiny
stačí, pokud se budou rovnat jejich postačující statistiky. Dokážeme jednoduše
spočítat první dva momenty Dirichletovského rozdělení, a proto použijeme
aproximaci a budeme počítat pouze s~nimi a zbylé momenty zanedbáme.
Je tedy třeba nalézt střední hodnotu a rozptyl proměnných z~$p^*(\vec\theta)$:

\begin{align}
\mathbb{E}_{p^*}[\vec\theta] &= \int \vec\theta \, p^*(\vec\theta) ~ \mathrm{d}\vec\theta
\\
&= \int \vec\theta \, (
	w_0^* \, Dir(\vec{\theta};\, \vec{\alpha}) +
	\sum_j \, w^*_j \,
        	Dir(\vec{\theta};\, \vec{\alpha} + \vec{\delta}_j)
    ) ~ \mathrm{d}\vec\theta
\\
&= w_0^* \int \vec\theta \, Dir(\vec{\theta};\, \vec{\alpha}) ~ d\vec\theta +
	\sum_j w_j^* \int \vec\theta \, Dir(\vec{\theta};\, \vec{\alpha} + \vec{\delta}_j)
    ~ \mathrm{d}\vec\theta
\\
&= w_0^* \, \mathbb{E}_{Dir(\vec\alpha)}[\vec\theta] +
	\sum_j \, w_j^* \, \mathbb{E}_{Dir(\vec\alpha + \vec{\delta}_j)}[\vec\theta]
\label{eq:epweimix}
\end{align}
Střední hodnotu proměnných $\vec\theta$ podle rozdělení $p^*$ lze podle (\ref{eq:epweimix}) spočítat
jako vážený součet středních hodnot $\vec\theta$ podle jednotlivých
Dirichletovských distribucí, z~kterých se $p^*$ skládá.

První moment tedy máme spočítaný, pro výpočet rozptylu můžeme využít
definici:
\begin{equation}
	Var[\vec{\theta}] = \mathbb{E}[\vec{\theta}^2] - \mathbb{E}[\vec\theta]^2.
\end{equation}

Chybí nám tedy ještě výpočet střední hodnoty druhé mocniny proměnné
$\vec\theta$ podle $p^*$. Můžeme ji vyjádřit z~definice střední hodnoty:

\begin{align}
\mathbb{E}_{p^*}[\vec\theta^2] &= \int \vec\theta^2 \, p^*(\vec\theta) ~ d\vec\theta
\\
&= w_0^* \int \vec\theta^2 \, Dir(\vec{\theta};\, \vec{\alpha}) ~ d\vec\theta +
	\sum_j w_j^* \int \vec\theta^2 \, Dir(\vec{\theta};\, \vec{\alpha} + \vec{\delta}_j)
	 ~ d\vec\theta
\\
&= w_0^* \, \mathbb{E}_{Dir(\vec\alpha)}[\vec\theta^2] +
	\sum_j w_j^* \, \mathbb{E}_{Dir(\vec\alpha + \vec{\delta}_j)}[\vec\theta^2].
\label{eq:epweimix2}
\end{align}

Opět získáváme vážený součet středních hodnot podle Dirichletovských rozdělení.
Střední hodnotu druhé mocniny proměnné podle Dirichletovského rozdělení
lze opět jednoduše odvodit z~definice:
\begin{align}
\mathbb{E}_{Dir(\vec\alpha)}[x_i^2] &=
	\int x_i^2 \, Dir(\vec x; \, \vec\alpha) \; \mathrm{d}\vec x
\\
&=
	\int x_i^2 \, \frac{\Gamma(\,\alpha_0\,)}
			   {\prod_{j=1}^N \, \Gamma(\,\alpha_j\,)} \;
		\prod_{j=1}^N x_j^{\alpha_j - 1} \; \mathrm{d}\vec x.
\end{align}

Nyní jsme v~podobné situaci jako v~(\ref{eq:4}). Budeme postupovat stejně,
pokusíme se výraz přepsat do formy Dirichletovské distribuce nad $x_i$:

\begin{align}
\mathbb{E}_{Dir(\vec\alpha)}[x_i^2] &=
\int \frac{\Gamma(\,\alpha_0 + 2\,) \; \alpha_i \; (\,\alpha_i + 1\,)}
		{\alpha_0 \; (\,\alpha_0 + 1\,) \; \Gamma(\,\alpha_i + 2\,) \; \prod_{j \ne i} \; \Gamma(\,\alpha_j\,)} \;
    x_i^{\alpha_i + 1} \, \prod_{j \ne i} \, x_j^{\alpha_j - 1} \;
\mathrm{d}\vec x
\\
&= \frac{\alpha_i \; (\,\alpha_i + 1\,)}
		{\alpha_0 \; (\,\alpha_0 + 1\,)} \,
	\int \frac{\Gamma(\,\beta_0\,)}
		{\prod_i \,\Gamma(\,\beta_i\,)} \;
        \prod_i \, x_i^{\beta_i - 1} \;
    \mathrm{d}\vec x
\\
&= \frac{\alpha_i \; (\,\alpha_i + 1\,)}
		{\alpha_0 \; (\,\alpha_0 + 1\,)}
	\int Dir(\vec x;\, \vec\beta)\;
    \mathrm{d}\vec x
\\
&= \frac{\alpha_i \; (\,\alpha_i + 1\,)}
		{\alpha_0 \; (\,\alpha_0 + 1\,)}.
\end{align}

Vyjádřili jsme $\Gamma(\,\alpha_0\,)$ a $\Gamma(\,\alpha_i\,)$ s~pomocí $\Gamma(\,\alpha_0 + 2\,)$ a $\Gamma(\,\alpha_i + 2\,)$:
\begin{align}
\Gamma(\alpha_0) = \frac{\Gamma(\,\alpha_0 + 2\,)}
				{\alpha_0\;(\,\alpha_0 + 1\,)},
\quad \quad
\Gamma(\alpha_i) = \frac{\Gamma(\,\alpha_i + 2\,)}
				{\alpha_i \; (\,\alpha_i + 1\,)}.
\end{align}

Následně jsme vytvořili nové parametry $\vec\beta$:
\begin{align}
\beta_i = \alpha_i + 2, \quad \quad
\beta_{j \ne i} = \alpha_j, \quad \quad
\beta_0 = \sum_i \beta_i.
\end{align}

Nyní tedy dokážeme spočítat $\mathbb{E}_{p^*}[\vec\theta]$ a $\mathbb{E}_{p^*}[\vec\theta^2]$.
Parametry aproximovaného rozdělení nalezneme následovně
\begin{align}
\frac{\mathbb{E}[X_1] - \mathbb{E}[X_1^2]}
     {\mathbb{E}[X_1^2] - \mathbb{E}[X_1]^2} &=
\frac{
	\frac{\alpha_1}
		{\alpha_0}
	- \frac{\alpha_1(\alpha_1 + 1)}
		{\alpha_0(\alpha_0 + 1)}
	}
	{
	\frac{\alpha_1(\alpha_1 + 1)}
		{\alpha_0(\alpha_0 + 1)}
	- \frac{\alpha_1^2}
		{\alpha_0^2}
	}
\label{eq:5}
\\
&=
\frac{
    \frac{\alpha_1 (\alpha_0 + 1) - \alpha_1(\alpha_1 + 1)}
         {\alpha_0 (\alpha_0 + 1)}
}{
    \frac{\alpha_0 \alpha_1 (\alpha_1 + 1) - \alpha_1^2 (\alpha_0 + 1)}
         {\alpha_0^2 (\alpha_0 + 1)}
}
\\
&=
\frac{\alpha_0 \alpha_1 (\alpha_0 - \alpha_1)}
     {\alpha_1 (\alpha_0 \alpha_1 + \alpha_0 - \alpha_0 \alpha_1 - \alpha_1)}
\\
&=
\alpha_0
\\
\alpha_i &= \mathbb{E}[X_i] \;\alpha_0 \label{eq:6}
\end{align}

Z~rovnice~(\ref{eq:5}) vypočítáme sumu všech parametrů $\alpha_0$. 
Protože střední hodnota proměnné z~Dirichletovského rozdělení je právě
$\frac{\alpha_i}{\alpha_0}$, tak jednotlivé parametry získáme z~rovnice~(\ref{eq:6}).

\section{Algoritmus}

V~algoritmu~\ref{alg:ep} je předveden algoritmus Expectation Propagation upravený pro potřeby učení parametrů.

\begin{algorithm}
\caption{Expectation Propagation pro učení parametrů}
\label{alg:ep}
\begin{algorithmic}
\State Parametry zpráv z~faktoru $\beta$ do množiny parametrů
    $\vec{\theta}_i$ označíme $\vec\alpha_{f_\beta \rightarrow \vec\theta_i}$.
\State Parametry zpráv z~množiny parametrů $\vec{\theta}_i$ do faktoru $\beta$
    označíme $\vec\alpha_{ \vec\theta_i \rightarrow f_\beta}$.
\State Parametry marginální distribuce množiny parametrů $\vec{\theta}_i$
    označíme $\vec{\alpha}_i$.
\Init
    \State Nastav zprávy mezi faktory a proměnnými na $1$.
    \State Nastav parametry $\vec\alpha_{f_\beta \rightarrow \vec\theta_i}$ na $1$.
    \State Nastav parametry $\vec{\alpha}_i$ na apriorní hodnotu.
\EndInit
\Repeat
    \State Vyber faktor $f_{\tilde\beta}$, který se bude aktualizovat.

    \State Spočítej všechny zprávy z~parametrů:
    \For{každý parametr $\vec{\theta}_i$ spojený s~faktorem $f_{\tilde\beta}$}
        \State Parametry zprávy z~$\vec{\theta}_i$ do $f_{\tilde\beta}$:
            $\vec\alpha_{ \vec\theta_i \rightarrow f_{\tilde\beta}} = \vec{\alpha}_i -
            \vec\alpha_{f_{\tilde\beta} \rightarrow \vec\theta_i } + 1$.
    \EndFor

    \State Aktualizuj zprávy z~faktoru do proměnných:
    \For{každou proměnnou $X_i$, spojenou s~faktorem $f_{\tilde\beta}$}
    \State Zpráva z~$f_{\tilde\beta}$ do $X_i$ podle (\ref{eq:msgfromftox}):
    \State
        \hskip\algorithmicindent
        $m_{f_{\tilde{\beta}} \rightarrow x_j} =
            \sum_{\vec{x} \backslash x_j}
                \mathbb{E}_{q^{\backslash\tilde\beta}}(\,\theta_{\rho(\vec{x^\prime}), x_0}\,)\,
                \prod_{i \ne j}\,
                    q^{\backslash\tilde\beta}(x_i)$
    \EndFor

    \State Aktualizuj marginální pravděpodobnost parametrů:
    \For{každý parametr $\vec{\theta}_i$, spojený s~faktorem $f_{\tilde\beta}$}
        \State Spočítej parametry $\vec{\alpha}_i^*$ pro Dirichletovské
            rozdělení, které nejlépe
        \State aproximuje cílovou marginální distribuci
            (\ref{eq:marginaltheta}). Metoda popsána
        \State v~předchozí sekci.
        \State Parametry zprávy z~$f_{\tilde\beta}$ do $\vec{\theta}_i$:
        \State
            \hskip\algorithmicindent
            $\vec\alpha_{f_{\tilde\beta} \rightarrow \vec\theta_i} =
                \vec{\alpha}_i^*
                - \vec\alpha_{ \vec\theta_i \rightarrow f_{\tilde\beta}}
                + 1$
        \State Aktualizuj parametry marginální distribuce $q(\vec{\theta}_i)$
        \State
            \hskip\algorithmicindent
            $\vec{\alpha}_i = \vec{\alpha}_i^* =
                \vec\alpha_{f_{\tilde\beta} \rightarrow \vec\theta_i}
                + \vec\alpha_{ \vec\theta_i \rightarrow f_{\tilde\beta}}$
    \EndFor
    \For{každou proměnnou $X_i$, spojenou s~faktorem $f_{\tilde\beta}$}
        \State Aktualizuj zprávy z~proměnných do faktoru:
        \State
            \hskip\algorithmicindent
            $q^{\backslash\tilde\beta}(x_i) =
                \prod_{\beta \ne \tilde\beta} \, m_{f_{\beta} \rightarrow x_i}(x_i)$
    \EndFor
\Until{konvergence}
\end{algorithmic}
\end{algorithm}


% Ukázka použití některých konstrukcí LateXu (odkomentujte, chcete-li)
% \include{ukazka}

% !TEX root = prace.tex
\chapter*{Závěr}
\addcontentsline{toc}{chapter}{Závěr}

Práce splnila všechny vytyčené cíle.
Výsledkem je implementace metod pro odhad stavu a parametrů v dialogových systémech.
Byly představeny metody pro inferenci v grafických modelech založené na posílání zpráv, navíc byl odvozen algoritmus pro inferenci parametrů dirichletovského rozdělení srovnáním momentů.
Práce obsahuje příklady použití vytvořené knihovny a otestování na reálných datech v rámci soutěže Dialog State Tracking Challenge, kde generativní systém pro odhad stavu používající implementovanou knihovnu dosáhl výsledků srovnatelných s ostatními přihlášenými týmy, které se dají považovat za state of the art.
Bližší detaily podle cílů vytyčených v úvodu práce:
\begin{enumerate}

\item V kapitole \ref{ch:kap1} byly představeny dialogové systémy a jejich jednotlivé součásti.
V sekci \ref{sec:bn} pak byly představeny bayesovské sítě, které se nabízí jako ideální model pro reprezentaci dialogového stavu.

\item V kapitole \ref{ch:kap2} jsou prezentovány možné inferenční algoritmy pro odhad stavu v bayesovských sítích.
V sekci \ref{sec:lbp} je představen algoritmus Loopy Belief Propagation.
Jeho použití je ukázáno v sekci \ref{sec:usage}.

\item Implementace LBP byla použita pro DSTC a popis celé soutěže i s výsledky je v sekci \ref{sec:dstc}.
Implementovaný systém pro odhad stavu byl lepší než baseline systém poskytnutý organizátory (tabulka \ref{t:all:datasets}) a systém se navíc umístil příznivě i vzhledem k ostatním přihlášeným týmům (tabulka \ref{t:DSTC:ranking}).

\item Implementované strategie pro výběr vrcholu v LBP algoritmu jsou popsány v sekci \ref{sec:noch}.

\item Algoritmus Expectation Propagation pro inferenci v obecných grafických modelech je představen v sekci \ref{sec:ep}.

\item Knihovna a jednotlivé její součásti jsou popsány v kapitole \ref{ch:kap4}.
Příklad vrcholů vytvořených speciálně pro použití v EP je v sekci \ref{sec:vrdir}.
Učení parametrů pro odhad stavu v dialogových systémech je odvozeno v kapitole \ref{ch:ep}.
\end{enumerate}

\singlespacing

%%% Seznam použité literatury
\addcontentsline{toc}{chapter}{Seznam použité literatury}
\bibliography{citace}{}
\bibliographystyle{czech}

%%% Tabulky v diplomové práci, existují-li.
\chapwithtoc{Seznam tabulek}

%%% Použité zkratky v diplomové práci, existují-li, včetně jejich vysvětlení.
\chapwithtoc{Seznam použitých zkratek}

%%% Přílohy k diplomové práci, existují-li (různé dodatky jako výpisy programů,
%%% diagramy apod.). Každá příloha musí být alespoň jednou odkazována z vlastního
%%% textu práce. Přílohy se číslují.
\chapwithtoc{Přílohy}

\openright
\end{document}
