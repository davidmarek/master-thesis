\chapter*{Úvod}
\addcontentsline{toc}{chapter}{Úvod}

Dialog je přirozený způsob dorozumívání a sdělování informací mezi lidmi.
Počítač, který by dokázal vést dialog s uživatelem, byl vždy snem nejen
příznivců vědecko-fantastické literatury.
Už pro první počítače vnikaly programy, které se snažily využívat přirozenou
řeč pro interakci s uživatelem.
Jedním z takových programů byl například Eliza, program, který předstíral, že
jej zajímá, co mu uživatel říká.
Fungoval na principu rozpoznání textu pomocí gramatiky a následné transformace
textu do promluv dle pravidel.
Avšak gramatiky a pravidlové systémy se ukázaly nedostačné pro praktické
aplikace a tak se vývoj přesunul do statistických metod.
S využitím statistickým metod a metod strojového učení bylo možné začít s
porozumíváním mluveného slova.
Přijetí bylo zpočátku chladné a veřejnost byla 
