% !TEX root = prace.tex
\chapter*{Úvod}
\addcontentsline{toc}{chapter}{Úvod}

Dialog je přirozený způsob dorozumívání a sdělování informací mezi lidmi.
Počítač, který by dokázal vést dialog s uživatelem, byl vždy snem nejen
příznivců vědecko-fantastické literatury.
Už pro první počítače vnikaly programy, které se snažily využívat přirozenou
řeč pro interakci s uživatelem.
Jedním z takových programů byl například Eliza, program, který předstíral, že
jej zajímá, co mu uživatel říká.
Fungoval na principu rozpoznání textu pomocí gramatiky a následné transformace
textu do promluv dle pravidel.
Avšak gramatiky a pravidlové systémy se ukázaly nedostačné pro praktické
aplikace a tak se vývoj přesunul do statistických metod.
S využitím statistickým metod a metod strojového učení bylo možné začít s
porozumíváním mluveného slova.
Přijetí bylo zpočátku chladné a veřejnost byla 

\section*{Cíle}

\begin{enumerate}
\item Dialogové systémy často využívají pouze nejlepší hypotézu ze systému porozumění přirozené řeči.
    Většina ovšem umí vytvořit seznam $n$ nejlepších hypotéz.
    Tato práce si klade za cíl představit metody pro inferenci dialogového stavu v dialogovém systému s využitím více hypotéz.
    Představené metody budou založeny na reprezentaci dialogového stavu pomocí dynamických bayesovských sítí.
\item Bude představen algoritmus Loopy Belief Propagation, který bude implementován pro využití v reálných systémech pro odhad dialogového stavu.
\item Algoritmus bude otestován na datech z reálného dialogového systému Let's Go a porovnán s dalšími systémy, které se účastnily soutěže Dialog State Tracking Challenge 2013.
\item Důležitou části algoritmů pro inferenci je určení pořadí, v jakém má inference probíhat.
    Práce bude obsahovat implementaci několika strategií pro inferenci, které budou umožňovat efektivní inferenci pro různé druhy bayesovských sítí (stromy, dynamické sítě, obecné grafy).
\item Nakonec se práce bude zabývat učením parametrů sítě a představí algoritmus Expectation Propagation, který umožňuje inferenci v bayesovských sítích se spojitými náhodnými proměnnými.
\item Při většině reálných použití dochází k aproximacím a úpravám modelu v závislosti na problému tak, aby bylo možné inferenci provádět v reálném čase.
Práce bude obsahovat implementaci frameworku, do kterého je možné jednoduše zasadit vlastní moduly pro aproximaci pravděpodobnostních rozdělení, které bude algoritmus Expectation Propagation používat.
\item Jako příklad bude ukázán systém pro učení parametrů pravděpodobnostního rozložení pro pozorování v generickém modelu pro reprezentaci dialogového stavu.
\end{enumerate}
