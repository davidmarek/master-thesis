% !TEX root = prace.tex
\chapter{Bayesovské sítě a inference}

\section{Bayesovské sítě}

Bayesovské sítě jsou pravděpodobnostní grafický model, který využívá podmíněných nezávislostí pro úspornou reprezentaci sdružené pravděpodobnosti.
Bayesovská sít je orientovaný acyklický graf, jeho vrcholy jsou náhodné proměnné a hrany odpovídají přímé závislosti jednoho uzlu na druhý.
Pro každou náhodnou proměnnou v síti platí, že její pravděpodobnost je jednoznačně určena jejími rodiči v grafu.
Podmíněná pravděpodobnostní distribuce (CPD) proměnné $X$ popisuje pravděpodobnost proměnné $X$ dáno její rodiče, $P(X \mid parents(X))$.
Pokud proměnná nemá žádná rodiče, pak její podmíněná pravděpodobnostní distribuce je ekvivalentní marginální pravděpodobnostní distribuci.

Příklad Student~\cite{koller2009probabilistic}: firma zvažuje, zda-li přijme studenta.
Firma chce přijímat chytré studenty, ale nesmí je testovat na inteligenci (I) přímo.
Má však výsledek studentových SAT testů, které ale nemusí stačit pro správné zhodnocení inteligence.
Požadují tak tedy i doporučení (D) od jednoho z učitelů.
Učitel studentovi napíše doporučující dopis na základě známky (Z), kterou student získal v jeho předmětu.
Předměty se ovšem liší v obtížnosti (O) a tak je studentova známka v předmětu závislá nejen na jeho inteligenci, ale také na obtížnosti předmětu.
Grafický model reprezentující tento problém je vyobrazen na obrázku~\ref{fig:student}.
\begin{figure}
\begin{center}
\begin{tikzpicture}

\matrix[row sep=0.75cm, column sep=1.2cm]
{
    \node[latent]       (O)     {O};
    && \node[latent]    (I)     {I};
    \\
    &\node[latent]      (Z)     {Z};
    &&\node[latent]     (S)     {SAT};
    \\
    &\node[latent]      (D)     {D};
    \\
};

\edge{O}{Z}
\edge{I}{Z}
\edge{I}{S}
\edge{Z}{D}

\end{tikzpicture}
\end{center}
\label{fig:student}
\caption{Bayesovská síť pro příklad se studentem.}
\end{figure}

V tomto modelu je několik nezávislostí. Obtížnost předmětu a inteligence studenta jsou zjevně nezávislé.
Studentova známka z předmětu je závislá na obtížnosti předmětu a inteligenci studenta, ale je podmíněně nezávislá na jeho výsledku ze SAT, dáno studentova inteligence.
Konečně doporučení, které student obdrží, je podmíněně nezávislé na všech ostatních proměnných, dáno studentova známka.

Sdruženou nezávislot tohoto modelu lze zapsat ve formě podmíněných pravděpodobnostních distribucí s pomocí řetízkového pravidla.
\begin{equation}
P(O, I, Z, S, D) = P(D \mid Z) P(Z \mid O, I) P(SAT \mid I) P(O) P(I)
\end{equation}

Předpokládejme, že obtížnost předmětu, inteligence studenta, doporučující dopis a výsledek SAT jsou binární proměnné.
Známka z předmětu pak je ternární proměnná.
Pokud bychom zapsali sdruženou pravděpodobnost ve formě tabulky, tak se dostaneme k 48 položkám.
Díky rozdělení do podmíněných pravděpodobnostních rozložení, které nám bayesovská síť poskytuje, se dostáváme k $2 + 2 + 12 + 4 + 6 = 26$ položkám.
Tedy i v tomto jednoduchém modelu dochází k značné úspoře.

\subsection{Šíření informace v bayesovských sítích}

\section{Inference v Bayesovských sítích}

\section{Aproximativní inference}
